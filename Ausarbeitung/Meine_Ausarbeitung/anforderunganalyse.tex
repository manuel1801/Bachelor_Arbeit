\chapter{Anforderungen und Analyse}\label{kap:anforderunganalyse}

\section{Ziel der Arbeit}\label{sec:zielderarbeit}

\begin{itemize}
    \item Tiere sollen erkannt und auch lokalisiert werden. -> bbox
    \item keine echtzeit, jedoch dennoch alle relevanten frames von camera input verarbeiten
    \item insbesondere nacht geeignet, da tiere oft nacht aktiv -> infrarotfähige kamera
    \item implementierung auf einem Raspberry Pi und Inferenz auf dem NCS2 mit hilfe OpenVino
    \item möglichkeit zur nutzer benachrichtigung bei erkennung zb per email und senden des Bildes
\end{itemize}
Erkennu g mithilfe CNNs. Lokalisierung entweder mit keras ggf merhstufig oder 
mit end to end lösung wie zb ssd oder f-rcnn
\\
End to end Prozess von Datensatz beschaffung, über training eines geeigneten Neuronalen 
Netzes bis hin zu implementierung der Applikation, 
die auf einem Raspberry Pi läuft und die Inferenz auf dem NCS2 ausführt.
\\
Es sollen in Wild Tiere erkannt werden, die in Deutschland heimisch sind.\\
Das system soll autonom laufen und den Nutzer informieren (und das erkannte bild senden) 
sobald etwas erkannt wurde.\\
Im Optimalfall soll es mithilfe Infrarot kamera auch im Dunkeln Tiere erkennen, 
(da Nachts mehr tiere zu sehen sein werden)


Verwender werden soll: für Inferenz der in \ref{sec:hardware} beschriebene
Neural Compute Stick 2, und für die Stuerung der (Einptaininen Computer) 
RaspberryPi2. 

Um auch im Dunklen oder bei nacht Tiere erkennen zu können soll eine Kamera ohne
Infrarot Filter verwendet werden. (evtl noch auf realsense eingehen)
\\
Die Kommunikation zwischen Raspberry und Pc soll über eine server/client tcp 
Verbindung erfolgen. Die Applikattio soll mitteilen wenn etwas erkannt wurde 
und das bild zusenden. Ausserdem soll das aktuelle frame abgefragt werden können
sowie einstellungen bezüglich infrarot leds vorgenommen werden können.






\section{Related Work}\label{sec:related_work}

hier obj det architecturen wie ssd und faster r-cnn und aktuelle trends

\begin{itemize}
    \item Gibt einen Überblick über verwandte Arbeiten im Gebiet
    \item Strukturiert und Gruppiert diese Arbeiten sinnvoll
    \item Deckt möglichst alle relevanten Arbeiten ab
    \item Erklärt kurz deren Inhalt und was sie von anderen Arbeiten (vorallem der Eigenen!) abheben
    \item Positioniert die eigene Arbeit im Gebiet
\end{itemize}
