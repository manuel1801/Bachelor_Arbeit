\chapter{Einleitung}\label{kap:einleitung}

Wildkameras werden für die Jagt oder zu biologischen 
Forschungszwecken eingesetzt.
Das Aufnehmen der Bilder erfolgt dabei meist über einen 
Bewegungsmelder automatsich.
Dadurch kann unter Umständen eine große menge 
an unwichtigen Daten entstehen, welche
speicherplatz, und, bei automatischem 
Senden, Datenvolumen benötigen, sowie ein großen 
Auswertungsaufwand mit sich bringen.

Ein System welches genau erkennt um welche Tiere es sich handelt,
kann gezielter für verschiedene Anwendungen 
wie beispielsweise das Aufspüren von Wolf und Bär im Wald,
den Fuchs im eigenen Garten oder um den Tierbestand seltener 
aussterbender Tierarten zu erfassen.

Die Umsetzung eine solchen Kamerasystems mit
 Erkennungsfunktion erfolgt meistens mithilfe  
 Deep Learning, ein Teilgebiet der Künslichen Intelligenz.

Die Fortschritte, die in den letzten Jahren in diesem 
Bereich gemacht wurden, sowie die verfügbare Hardware, ermöglichen 
die Entwicklung eines solchen Systems auch ohne Großrechner 
für den Privatgebrauch.

Ziel der vorliegenden Arbeit war es ein autonomes Kamerasystem 
zu entwickeln, welches mithilfe von Deep Learning Algorithmen, 
verschiedene Wildtierarten erkennen und klassifizieren kann.
 
Die Inferenz soll dabei auf dem KI Beschleuniger Neural Compute  
Stick 2 von Intel ausgeführt werden. Durch Verwendung einer  
infrarotfähigen Kamera soll es auch möglich sein 
bei Dunkelheit Tiere zu erkennen.

Die Anwendung wird über einen Raspberry Pi 4 laufen 
und selbständig Bilder erkannter Tieren an den Nutzer senden.

Damit gliedert sich die Arbeit zunächst in ein Grundlagen Kapitel 
welches sich mit Künslichen Neuronalen Netzen für die Bilderkennung 
sowie der verwendeten Harweare auseinander setzt.

Anschiließend wird es um das Training und die 
Auswertung geeigneter Modelle zur Objekterkennung gehen.

Das letzte Kapitel beschreibt die Entwicklung der Anwendung 
in welcher die Inferenz implementiert.
