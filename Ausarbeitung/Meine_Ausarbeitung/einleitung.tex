\chapter{Einleitung}\label{kap:einleitung}

Im Rahmen der Bachelor Arbeit wurde ein Überwachungssystem zur Wildtiererkennung,
entwickelt, welches auf einem Raspberry Pi läuft und den Nutzer bei Erkennung
bestimmter Tiere automatisch benachrichtigt, sowie das Bild and einen Server sendet.

Die Erkennung der Tiere erfolgte mithilfe Neuronaler Netze, wodurch es möglich ist 
die Überwachung gezielt nur auf bestimmte, relevante Tiere anzuwenden und so den
Datenverkehr gering zu halten. 

Die Inferenz der Neuronalen Netze wurde dabei auf einer separaten Hardware, dem 
Neural Compute Stick 2 von Intel ausgeführt.

Des weiteren wurde eine Infrarotfähige Kamera verwendet, damit das System auch
in der Nacht einsetzbar ist.
