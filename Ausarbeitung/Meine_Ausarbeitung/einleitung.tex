\chapter{Einleitung}\label{kap:einleitung}

Im Rahmen der Bachelor Arbeit wurde ein Überwachungssystem zur Wildtiererkennung,
entwickelt, welches auf einem Raspberry Pi läuft und den Nutzer bei Erkennung
bestimmter Tiere automatisch benachrichtigt, sowie das Bild and einen Server sendet.

Die Erkennung der Tiere erfolgte mithilfe Neuronaler Netze, wodurch es möglich ist 
die Überwachung gezielt nur auf bestimmte, relevante Tiere anzuwenden und so den
Datenverkehr gering zu halten. 

Die Inferenz der Neuronalen Netze wurde dabei auf einer separaten Hardware, dem 
Neural Compute Stick 2 von Intel ausgeführt.

Des weiteren wurde eine Infrarotfähige Kamera verwendet, damit das System auch
in der Nacht einsetzbar ist.


\textbf{aus ziel der Arbeit}



Wie in der Einleitung \ref{kap:einleitung} beschrieben, soll 
ein CNN Basiertes System zur Wildtiererkennung entwickelt 
werden, das für dir Inferenz den Neural Compute Stick 2 verwendet.
Dabei sollte neben der reinen Erkennung auch eine Lokalisierung 
der erkannten Tiere im Bild stattfinden. Gängige Techniken dafür 
werden im nächsten Abschnitt erläutert.
\\
Dabei soll das Deep Learing Modell im Rahmen der gegebenen 
möglichkeiten und Limitierungen der Hardware möglichst 
genau und Robust sein, sodass es auch für die graustufen 
Bilder der Infrarot Kamera zuverlässig funktioniert.
Da eine erhöht Genauigkeit auch immer mit einer größeren
Latenz für der Inferenzzeit einhergeht war dies ein mit 
zu berücksichtigender Punkt.
\\
Neben training und evaluierung eines geeigneten Deep Learning Modells,
war die Implementierung der Anwendung, welche die Inferenz 
des Modells ausführt ein weiterer Bestandteil der Arbeit.
\\
Diese soll voll autonom auf dem Raspberry Pi laufen,
über eine mobile Netzwerk Verbindung verfügen und 
mittels eines geeigneten Kommunikations Protokolls die 
die erkannten und abgespeicherten Bilder an einen
Heim Pc senden.
Des weiteren sollte eine geeignete Kamera verwendet werden, die 
sowohl normale, als auch Infrarot Aufnahmen machen kann.
