\chapter{Realisierung Objekt Erkennung}\label{kap:objerk}

\section{Dataset}\label{sec:dataset}


Da es sich um supervised Learning handelt müssen trainigs daten gelabelt werden.\\
für validierung und test muss datensatz zu 80, 10, 10 in test/train/validation aufgeteilt werden.
\\
wie in \ref{sec:ml} beschieben dient das validierungs set zur überwachung wärend des trainings für overfitting.\\
Mit dem Test set kann nach dem training die Inferenz also das ausfüren des traininierten models getestet werden.

Für die Objekterkennung muss der Datensatz wie in \ref{sec:deepl_cv} beschrieben neben den gelabelten bildern auch
die X- und Y-Koordinaten der Bounding Boxen, welche das objekt auf dem Bild umramt, enthalte. das Objekt befindet enthalten.\\


\subsection{Datenbeschaffung}\label{subsec:get_data}

Da das Erstellen eines eingangs erwähnten Datensets von Hand sehr müsam ist wird meistens auf Quellen zurückgegriffen, die
schon gelabelte Daten zu bestimmten Klassen zur verfügung stellen. Neben den in \ref{subsec:comp} vorgestellten Seiten
bietet OpenImages einen vielzahl an Klassen an, darunter auch Unter der Kategorie Säugetiere eine Auswahl an Wild Tier, welche 
im folgenden verwendet wurde.

Mit einem Open source Tool \cite{OpenImages} konnten eine teilmenge aus dem gesammten Open Images datensatzes herunter geladen werden.

Die Label Files haben das anotierungs format \texttt{class,xmin,ymin,xmax,ymax} welches wie in \ref{subsec:tfrecord} 
beschrieben wird, noch in ein für tensorflow vertändliches format gebracht werden musste.

Die Verteilung der Klassen im Datensatz war nicht ausbalanciert, wie in \ref{fid:distribution} zu sehen ist.

Das kann zur folge haben das.\\

Allg wie viele samples sollte man haben.\\

Augmentierung im folgenen Teil beschieben.



\subsection{Augmentierung}




was ist Augmentierung\\
wie wurde es angewand\\
bsp bilder



\subsection{TF Record Files}\label{subsec:tfrecord}

was es ist\\
wie es erstellu wurde\\

Wie in \ref{subsec:get_data} erwähnt verwendet tensorflow ein bestimmtes format für das datenset, sog 
Protocol buffer tf record files, dateien im Binary format die sowohl die Bilder als 
auch die Labels enthalten. das sind Protocol buffer welche die daten serealisieren.\\
evtl hier besp ausschnitt von aufbau eines proto elements.\\

Um nun die von OpenImages herunter geladenen Bilder und Label Files in das TFRecords Format zu bringen waren 
mehrere Schritte nötig.

OI - VOC - csv - tf.records



\section{Training}

\subsection{TF obj det api}

Für das Training wurde das Framework Tensorflow verwendet, welces eine Api für Objekterkennung bietet. 

welche pretrained modell gibt es und welche kamen in frage (für ncs2)
Die Tensorflow Objekt Detection Api bietet eine vielzahle an vortrainierten Modellen, 
dabei wurden die meiseten auf den COCO datensatz trainiert.\\

speed/acc trade off \\%\cite{paper speed/acc}

daneben war bei der bei der Ausahl die kompatibilität zu Open Vino zu berücksichtigen:\\
liste von kompatiblen modellen.\\

trainiert wurde auf:\\
\begin{itemize}
    \item ssd mobile net und inception
    \item faster rcnn (inception und resnet)
    \item frcnn
\end{itemize}

(in eval ergebniss dann etwa so: ssd zu schlechte 
performence und für appl keine realtime nitwendig, 
frcnn zu langsam, faster rcnn gute mitte)

hier ersten durchlauf (mit Overfitting) darstellen

\subsection{Regularisierung}

Um das Overfitting zu vermeiden gibt es wie in \ref{sec:nn} 
beschrieben vershiedene Möglichkeiten.

Untersucht wurde hier

\begin{itemize}
    \item Augmentierung
    \item Early Stopping
    \item \dots weitere zB $L_{1}$, $L_{2}$
\end{itemize}



\subsection{Training grayscale}\label{subsec:train_gray}

Da die Kamera im Infrarot Modus ein Graustufen Bild mit nur einem 
Farbchannel liefert, muss dies für die Inferenz berücksichtigt werden.

Es ergeben sich hier mehrere möglichkeiten:

\begin{enumerate}
    \item Normales (r, g, b) Netz
    \item Ein Farbchannel (gr) Netz
    \item Drei Farbchannel (gr, gr, gr)
\end{enumerate}

Für 1. und 3. Müssen die bilder der Kamera vor der 
Inferenz auf 3 Farbchannel ($3 \times grau$) erweitert werden.
\\
Um das Netz auf einen Channel zu trainieren wurde im config file \dots
\\
Um $3 \times gray$ zu trainieren wurden die Bilder in OpenCV in 
grau convertiert und wieder als jpgs abgespeichert.

Die Ergebnisse sind in Kapitel \ref{kap:eval} dargestellt.


\section{Parameter Optimierung}
einstellungen im Config File\\
tensorflow graph oder plot zeigen\\
loss erklären (mit formel und für train und eval)


