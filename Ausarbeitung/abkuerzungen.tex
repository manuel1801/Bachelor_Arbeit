
\newglossaryentry{ml} 
{
    name={Machine Learning},
    description={Ein Bereich der künstlichen
    Intelligenz, der sich mit selbstlernenden 
    Algorithmen befasst},
    plural={Machine Learnings}
}

\newglossaryentry{nn} 
{
    name={Künstliche Neuronale Netze},
    description={Eine Form des Machine Learning, bei der 
    eine Vielzahl an künstlicher Neuronen miteinander 
    verbunden sind, um aus gegebenen Eingaben 
    die richtige Ausgabe zu erhalten},
    plural={künstlichen Neuronalen Netzen}
}

\newglossaryentry{cnn} 
{
    name={CNN (Convolutional Neural Network)},
    description={Neuronales Netz, welches math. Faltung 
    Verwendet, meist in der Bilderkennung verwendet.},
    first={Convolutional Neural Network (CNN)},
    text={CNN},
    plural={CNNs},
    firstplural={Convolutional Neural Networks (CNNs)}
}
\newglossaryentry{featuremap} 
{
    name={Feature Map},
    description={eine zweidimensionale matrix, 
    die duch die faltung eines bild mit einer 
    filtermatrix entsteht},
    plural={Feature Maps}
}
\newglossaryentry{downsampling} 
{
    name={Downsampling},
    description={In der Bildverarbeitung: kleiner skallieren 
    eines Bildes}
}

\newglossaryentry{zeropadding} 
{
    name={Zero Padding},
    description={Auffüllen mit nullen}
}

\newglossaryentry{overfitting}{
    name={Overfitting},
    description={Überanpassung eines Machine Learning Modells 
    an die Trainingsdaten, wodurch keine Generalisierung 
    mehr erreicht wird},
    plural={Overfittings}
}

\newglossaryentry{framework}{
    name={Framework},
    description={Programmgerüst, bestehend aus 
    mehreren Programmen, Schnittstellen, Tools, 
    und weiterem, zur Erleichterten Software 
    Entwicklung},
    plural={Frameworks}
}

\newglossaryentry{vm} 
{
    name={VM (Virtual Machine)},
    description={Virtuelle Nachbildung einer Rechnerarchitekture, 
    welche innerhalb eines lauffähigen Rechnersystems ausgeführt wird},
    first={Virtual Machine (VM)},
    text={VM}
}

\newglossaryentry{api} 
{
    name={API (Application Programming Interface)},
    description={Eine Application Programming Interface
    ist ein bestimmter satz an Regeln
    über die ein program funktionalitäten anderer software verweden kann.},
    first={Application Programming Interface (API)},
    text={API}
}
\newglossaryentry{ssh} 
{
    name={SSH (Secure Shell Protocoll)},
    description={Kommunikationsprotokoll},
    first={Secure Shell Protocoll (SSH)},
    text={SSH}
}
\newglossaryentry{scp} 
{
    name={SCP (Secure Copy Protocoll)},
    description={zum senden von daten über SSH},
    first={Secure Copy Protocoll (SCP)},
    text={SCP}
}
\newglossaryentry{smtp} 
{
    name={SMTP (Smart Mail Transfer Protokoll)},
    description={zum senden von emails},
    first={Smart Mail Transfer Protokoll (SMTP)},
    text={SMTP}
}

\newglossaryentry{thread} 
{
    name={Thred},
    description={wie prozess},
    plural={Threads}
}

\newglossaryentry{fps} 
{
    name={Fps (Frames per Second)},
    description={Bilder, die pro Sekunde verarbeitet werden können},
    first={Frames per Second (Fps)},
    text={Fps}
}
\newglossaryentry{csi} 
{
    name={CSI (Camera Serial Interface)},
    description={Kameraschnittstelle des Raspberry Pi's welche 
    ein Flachbandkabel verwendet},
    first={Camera Serial Interface (CSI)},
    text={CSI}
}

\newglossaryentry{soc} 
{
    name={SoC (System on Chip)},
    description={komplexes System, bestehend aud 
    CPI, GPU und weiterem auf einem Chip verbaut.},
    first={System on Chip},
    text={SoC}
}

\newglossaryentry{cpu} 
{
    name={CPU (Central Processing Unit)},
    description={zentrale Recheneinheit eines Computers},
    first={Central Processing Unit (CPU)},
    text={CPU}
}
\newglossaryentry{gpu} 
{
    name={GPU (Graphics Processing Unit)},
    description={Prozessor, der auf die Berechnungen von 
    Grafiken spezialisiert ist},
    first={Graphics Processing Unit (GPU)},
    text={GPU}
}
\newglossaryentry{vpu} 
{
    name={VPU (Vision Processing Unit)},
    description={Mikroprozessor für Bildverarbeitungsaufgaben 
    häufig in KI-Beschleunigern eingesetzt},
    first={Vision Processing Unit (VPU)},
    text={VPU}
}
