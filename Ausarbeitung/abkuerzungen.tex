
\newglossaryentry{ml} 
{
    name={Machine Learning},
    description={Ein Teilgebiet der Künstlichen
    Intelligenz, dass sich mit selbstlernenden 
    Algorithmen befasst. Damit können ohne explizite  
    Programmierung Zusammenhänge in größeren Datenmengen
    erkannt werden},
    plural={Machine Learnings}
}

\newglossaryentry{nn} 
{
    name={Künstliche Neuronale Netze},
    description={Eine Form des Machine Learnings, bei der 
    eine Vielzahl künstlicher Neuronen miteinander 
    verbunden sind. Indem die Verbindungen 
    unterschiedlich stark gewichtet sind, lassen 
    sich für gegebene Eingaben die richtigen 
    Ausgaben finden},
    plural={künstlichen Neuronalen Netzen}
}

\newglossaryentry{cnn} 
{
    name={CNN (Convolutional Neural Network)},
    description={Neuronales Netz, welches in 
    sog. \textit{Convolutional Layern} eine mathematische 
    Faltung des Inputbilds mit einer \textit{Filter-Matrix}
    durchführt},
    first={Convolutional Neural Network (CNN)},
    text={CNN},
    plural={CNNs},
    firstplural={Convolutional Neural Networks (CNNs)}
}


\newglossaryentry{rpn} 
{
    name={RPN (Region Proposial Network)},
    description={Ein CNN-basiertes Modell, dass 
    im \textit{Sliding-Window}-Verfahren aus den 
    Feature Maps räumliche Vorschläge für Objekte in
    einem Inputbild generiert},
    first={Region Proposial Network (RPN)},
    text={RPN},
    plural={RPNs},
    firstplural={Region Proposial Networks  (RPNs)}
}


\newglossaryentry{faltung}
{
    name={Faltung},
    description={Mathematische Rechenoperation, bei der 
    zwei Signale (z.b. in Form von Matrizen) übereinander
    verschoben und an jeder Stelle miteinander
    multipliziert werden. Anschließend 
    werden die Produktterme aufsummiert},
    plural={Faltungen}
}

\newglossaryentry{inferenz} 
{
    name={Inferenz},
    description={Die Anwendung 
    eines trainierten Machine-Learning-Modells
    für neue Inputdaten},
    plural={Inferenzen}
}

\newglossaryentry{featuremap} 
{
    name={Feature Map},
    description={Eine zweidimensionale Matrix, 
    die durch die Faltung eines Inputbilds mit einer 
    Filtermatrix entsteht},
    plural={Feature Maps}
}
\newglossaryentry{downsampling} 
{
    name={Downsampling},
    description={Reduktion einer Anordnung von zeitdiskreten
     Werten. In der Bildverarbeitung wird durch eine Verringerung 
     der Bildpunkte die Auflösung reduziert}
}

\newglossaryentry{zeropadding} 
{
    name={Zero Padding},
    description={Bei der Verrechnung zweier ungleich 
    großer Matrizen werden fehlende Werte mit Nullen 
    aufgefüllt, um die Ausgangsgröße für das 
    Ergebnis beizubehalten}
}

\newglossaryentry{overfitting}{
    name={Overfitting},
    description={Überanpassung eines Machine Learning Modells 
    an die Trainingsdaten, wodurch keine Generalisierung 
    für neue Daten mehr stattfinden kann},
    plural={Overfittings}
}

\newglossaryentry{framework}{
    name={Framework},
    description={Programmgerüst, dass die Entwicklung einer 
    Anwendung durch vorgefertigte Strukturen unterstützt,
    selbst aber kein vollständiges Programm darstellt},
    plural={Frameworks}
}

\newglossaryentry{vm} 
{
    name={VM (Virtual Machine)},
    description={Virtuelle Nachbildung einer Rechnerarchitektur, 
    welche innerhalb eines lauffähigen Rechnersystems
    ausgeführt wird},
    first={Virtual Machine (VM)},
    text={VM}
}

\newglossaryentry{api} 
{
    name={API (Application Programming Interface)},
    description={Eine Programmierschnittstelle, 
    mit der über einen definierten Satz an Regeln 
    auf die Funkionalitäten eines anderen Programms 
    oder eine Datenbank zugegriffen werden kann},
    first={Application Programming Interface (API)},
    text={API}
}
\newglossaryentry{ssh} 
{
    name={SSH (Secure Shell Protocoll)},
    description={Netzwerkprotokoll zur Herstellung
    einer verschlüsselten Netzwerkverbindung
    zu einem entfernten Gerät},
    first={Secure Shell Protocoll (SSH)},
    text={SSH}
}
\newglossaryentry{scp} 
{
    name={SCP (Secure Copy Protocoll)},
    description={Protokoll, welches über 
    eine SSH Verbindung die verschlüsselte
    Datenübertragung zwischen zwei Geräten 
    ermöglicht},
    first={Secure Copy Protocoll (SCP)},
    text={SCP}
}
\newglossaryentry{smtp} 
{
    name={SMTP (Simple Mail Transfer Protokoll)},
    description={Protokoll zum Senden von E-Mails},
    first={Smart Mail Transfer Protokoll (SMTP)},
    text={SMTP}
}

\newglossaryentry{thread} 
{
    name={Thread},
    description={Bezeichnet einen Ausführungsstrang 
    eines Computerprogramms und ist damit Bestandteil 
    eines Prozesses},
    plural={Threads}
}

\newglossaryentry{fps} 
{
    name={Fps (Frames per Second)},
    description={Frequenz, die angibt 
    wie viele Bilder pro Sekunde verarbeitet werden können},
    first={Frames per Second (Fps)},
    text={Fps}
}
\newglossaryentry{csi} 
{
    name={CSI (Camera Serial Interface)},
    description={Kameraschnittstelle des Raspberry Pi's,
    die ein Flachbandkabel verwendet},
    first={Camera Serial Interface (CSI)},
    text={CSI}
}

\newglossaryentry{soc} 
{
    name={SoC (System on Chip)},
    description={Komplexes System, bestehend aus 
    einer CPU, einer GPU und ggf. weiteren 
    Komponenten die gemeinsam auf einem Chip verbaut sind},
    first={System on Chip},
    text={SoC}
}

\newglossaryentry{cpu} 
{
    name={CPU (Central Processing Unit)},
    description={Zentrale Recheneinheit eines Computers},
    first={Central Processing Unit (CPU)},
    text={CPU}
}
\newglossaryentry{gpu} 
{
    name={GPU (Graphics Processing Unit)},
    description={Prozessor, der auf die Berechnung von 
    Grafiken spezialisiert ist},
    first={Graphics Processing Unit (GPU)},
    text={GPU}
}
\newglossaryentry{vpu} 
{
    name={VPU (Vision Processing Unit)},
    description={Mikroprozessor für Bildverarbeitungsaufgaben, 
    häufig in KI-Beschleunigern eingesetzt},
    first={Vision Processing Unit (VPU)},
    text={VPU}
}
