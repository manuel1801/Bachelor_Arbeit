\chapter{Einleitung}\label{kap:einleitung}


Wildkameras werden vielfach von Jägern genutzt, kommen
 zu Forschungszwecken zum Einsatz und werden auch
  zunehmend von Privatpersonen verwendet, die ihr eigenes
   Grundstück überwachen möchten.

Da das Aufnehmen der Bilder automatisch über einen
 Bewegungsmelder erfolgt, unabhängig davon, wodurch
  der Bewegungsmelder ausgelöst wird, kann es unter
   Umständen zu einer großen Menge an unwichtigen Daten
    kommen. Diese benötigen Speicherplatz auf dem Gerät
     oder verbrauchen bei automatischem Senden hohes
      Datenvolumen für eine mobile Netzwerkverbindung.
       Außerdem bringen sie einen großen Auswertungsaufwand
        mit sich, da nichtrelevante Daten ausgefiltert
         werden müssen.

Ein System, welches genau erkennt
 um welche Tiere es sich bei den gemachten
  Aufnahmen handelt, kann wesentlich effizienter 
  und gezielter für eine bestimmte Anwendung eingesetzt
   werden, beispielsweise zur Überwachung von
    Füchsen im eigenen Garten oder zum Aufspüren von
     Wölfen und Bären in bestimmten Waldgebieten.
      Auch der Artbestand seltener oder aussterbender
       Tierarten könnte so leichter erfasst werden.

Ziel der vorliegenden Arbeit ist es, ein autonomes
 Kamerasystem zu entwickeln, das mithilfe von Deep-Learning-Algorithmen
  verschiedene Wildtierarten erkennen und klassifizieren kann.
   Für die Umsetzung einer solchen Bilderkennungsaufgabe sind
    insbesondere Deep-Learning-Algorithmen geeignet,
     die ein Teilgebiet der künstlichen Intelligenz sind.
      Die Anwendung soll auf dem Einplatinencomputer
       \textit{Raspberry Pi 4} ausgeführt werden und die Bilder
        von erkannten Tieren automatisch an den Nutzer senden.
         Die Inferenz der Neuronalen Netze wird dabei auf dem
          KI Beschleuniger \textit{Neural Compute Stick 2} von \textit{Intel}
           ausgeführt. Durch Verwendung einer infrarotfähigen Kamera
            soll die Erkennung der Tiere auch bei Dunkelheit möglich sein.

Für die Bilderkennung werden zumeist \Glspl{cnn}
 verwendet. Durch die Fortschritte, die
  in diesem Bereich in den letzten Jahren gemacht wurden sowie
   durch die Verfügbarkeit leistungsfähiger und zugleich
    kostengünstiger Hardware ist die Realisierung einer
     solchen Anwendung auch für den Privatgebrauch und
      ohne Verwendung eines Großrechners möglich geworden.


Die Arbeit gliedert sich zunächst in ein Grundlagenkapitel,
 in dem die Funktionsweise von Künstlichen Neuronalen Netzen
  für die Bilderkennung behandelt wird. Anschließend
   geht es um die Umsetzung und Auswertung des Trainings
    geeigneter Deep-Learning-Modelle.
Der letzte Teil beschreibt die Entwicklung der
 Anwendung, in welcher die Inferenz eines
  fertig trainiertes Modells für den \textit{Neural
   Compute Stick 2} implementiert wird.


