\chapter{Einleitung}\label{kap:einleitung}

Ein \Gls{cnn} verwendet eine \Gls{featuremap} sowie 
eine \Gls{api}. damit kann das \Gls{cnn} \dots \Gls{featuremap}



Wildkameras kommen in der Jagt oder zu biologischen 
Forschungszwecken zum Einsatz.
Da das Aufnehmen der Bilder automatisch über einen 
Bewegungsmelder erfolgt, kann es unter Umständen zu 
einer großen Menge an unwichtigen Daten kommen.
Diese benötigen Speicherplatz auf dem Gerät, 
oder, bei automatischem Senden, Datenvolumen 
für die Mobile Netzwerkverbindung und bringen ausserdem einen 
großen Auswertungsaufwand mit sich.

Ein System, welches genau erkennt, um welche Tiere es
sich bei den gemachten Aufnahmen handelt, kann wesentlich 
effizienter und gezielter für eine bestimmte Anwendung 
eingesetzt werden.
Beispielsweise zur Überwachung des eigenen Gartens vor 
Füchsen, oder zum Aufspüren von Wölfen und Bären 
in bestimmten Waldgebieten.
Auch der Artbestand seltener oder aussterbender 
Tierarten könnte so leichter erfasst werden.

Für die Umsetzung einer solchen Bilderkennungsaufgabe 
werden Deep Learning Algorithmen verwendet, welche ein
Teilgebiet der künstlichen Intelligenz sind.
Bei den verwendeten Modellen handelt es sich in der Bilderkennung
meist um sogenannte Convolutional Neural Networks (CNNs).

Durch die Fortschritte, die in diesem Bereich in
den letzten jahren gemacht wurden, sowie durch 
die Verfügbarkeit leistungsfähiger und zugleich
kostengünstiger Hardware, ist die realisierung
einer solchen Anwendung auch für den Privatgebrauch
und ohne Verwendung eines Großrechners möglich geworden.


Ziel der vorliegenden Arbeit war es, ein autonomes Kamerasystem 
zu entwickeln, welches mithilfe von Deep Learning Algorithmen, 
verschiedene Wildtierarten erkennen und klassifizieren kann.
Dieses soll auf einem Raspberry Pi 4 laufen und 
Bilder von erkannten Tieren automatisch an den Nutzer 
senden.
Die Inferenz der Neuronalen Netze wird dabei
auf dem KI Beschleuniger Neural Compute Stick 2
von Intel ausgeführt werden. Durch verwendung einer 
infrarotfähigen Kamera soll die Erkennung des 
Systems auch bei Nacht stattfinden können.

Damit gliedert sich die Arbeit zunächst in ein
Grundlagen Kapitel, welches die funktionsweise 
von künstlichen Neuronalen Netzen für die
Bilderkennung behandelt.
Anschließend wrd es um die Umsetzung und Auswertung des 
Trainings geeigneter Deep Learning Modelle gehen.

Der letzte Teil beschreibt de Entwicklung der Anwendung, 
in welcher die Inferenz eines fertig trainiertes Modell, 
für den Neural Compute Stick, implementiert wird.