\chapter{Zusammenfassung und Ausblick}\label{kap:zusammenfassungausblick}


Ziel der Arbeit war, es ein autonomes
Kamerasystem zur Wildtierkennung,
mithilfe Neuronaler Netze, zu entwickeln.
Dafür wurden vortrainierte, CNN basierte, 
Deep Learning Modelle zur Objekterkennung
auf einen geeigneten Datensatz trainiert.
Dadurch sollte es Möglich sein, nicht 
nur die Anwesenheit eines Tieres 
zu Erkennen, sondern auch eine 
Klassifizierung der Tierart 
vorzunehmen, wodurch das System
gezielter für eine bestimmte
Anwendung eingesetzt werden kann.
Zur Realisierung, wurde neben einem Raspberry
Pi, sowie einer nachtsichtgeeigneten 
Kamera, für die Inferenz der 
\textit{Neural Compute Stick 2}
von \textit{Intel} verwendet, um die 
Verarbeitung der Daten auf dem 
Gerät ausführen zu können.
\vspace{0.5cm}

Für das Training wurde ein Datensatz, 
bestehend aus 9 Wildtierklassen verwendet,
welcher aus \textit{OpenImages} herunter 
geladen werden konnte.
Anschließend wurden die Daten 
für das Training aufbereitet und
zur Evaluierung, in verschiedene 
Sets aufgeteilt.
Das Training wurde dann mithilfe des 
Frameworks \textit{Tensorflow} durchgeführt,
wobei die Modelle \textit{SSD} und
\textit{Faster R-CNN} mit verschiedenen
Basis CNN und Parameter-Einstellungen
verwendet wurden.
Durch anschließende Evaluierung, konnte 
festgestellt werden, welches modell sich,
bezogen auf Genauigkeit und Geschwindigkeit,
am besten für die Anwendung eignet.
Der letzte Schritt war es die Inferenz, zusammen 
mit dem Anwendungscode für den Raspberry Pi 
zu implementieren, wofür mit OpenVino gearbeitet 
wurde.
\vspace{0.5cm}

Die Evaluierung der Modelle zeigte, dass 
eine erhöhte Genauigkeit, mit einer 
langsameren Inferenzzeit einhergeht.
Verbessert werden konnte die Genauigkeit 
zum einen durch eine Augmentierung der Daten 
was eine größere Robustheit gegenüber anderer 
Datensätze mitsich brachte, und zum 
anderen durch verwenden des Faster 
R-CNN, mit welchem auch Tiere weiter 
weg erkannt werden konnten.
Die Performance der Inferenz konnte durch 
asynchrone Inferenzausführung und 
verwenden eines bewegungsmelders sowie 
zwischenspeichern der frames verbessert werden.
\vspace{0.5cm}


Auffällig war der hohe Energieverbrauch, 
der durch den Neural Compute Stick, die Kamera mit 
Infrarot LEDs sowie dem Internetstick für den 
Raspberry Pi zustande kam.
Daher war kein langzeit testen der Anwendung möglich 
und ist bei fortführung der arbeit zu berücksichtigen.
ebenso wie eine geeignete hülle.
Verbesserunge der genauigkeit könnten durch
 training eines anderen datensatzes erreicht 
 werden, für effizientere inferenz könnte 
 ein anderes framework wie bsp caffe, welches 
 andere modelle enthällt zu verbesserungen führen.
 