\section[\thesection \  Applikation]{Applikation}\label{sec:application}

% Frame 12
\begin{frame}{Geschwindigkeit vs. Genauigkeit}
    Mit fertig trainierten Modellen kann die Inferenz auf test daten probeweise aungewendet ewrden\\
    zur bestimmung der Inferenz zeit.\\
    ergibt drawback
    \\
    tabelle von erg für modelle
    \\
    hier erklären mit ssd und faster rcnn architekturen    
\end{frame}

% Frame 13
\begin{frame}{Inferenz}
    model in ir format ins plugin auf hw geladen
    \\
    Inferenz: capture frame, preprocess, infer, postprocess
    \begin{itemize}
        \item sync
        \item async
    \end{itemize}
\end{frame}

% Frame 14
\begin{frame}{Umsetzung}
    anw fall tier erkennung: selten was da, aber wann dann alles wichtig
    \\
    daher: komplett asyncron, heist: frames werden in buffer geladen
    \\
    konzept in block diagramm:
    \\
    motion detection -> infer(parallel) -> senden an alle angemeldeten cients

\end{frame}

% Frame 16
\begin{frame}{Realworld Ergebnisse}

    hier inferierte bilder von endergebnis bei tag und nacht,\\
    nochwas zu infrarot bildern und rbg vs grau trainierte modell sagen.

\end{frame}
