\section[\thesection \  Applikation]{Applikation}\label{sec:application}

% Frame 13
\begin{frame}{Test Inferenz auf eigene Bilder} % ersetzen mit erste tests und vergl training auf gr bilder (inf erg von reh zeigen)
        
    \begin{columns}
        \column{0.5\columnwidth}
        \begin{figure}
            \centering
            \includegraphics[width=\textwidth]{Bilder/infer_result_ssd.jpg}
            \caption{SSD+InceptionV2}
        \end{figure}
        
        \column{0.5\columnwidth}
        \begin{figure}
            \centering
            \includegraphics[width=\textwidth]{Bilder/infer_result_faster.jpg}
            \caption{Faster R-CNN+InceptionV2}
        \end{figure}
    \end{columns}

\end{frame}

% Frame 14 (AsyncInf und Appl Ablauf in ein frame)
\begin{frame}{Applikation}

    % Define block styles
\tikzstyle{decision} = [diamond, draw, fill=blue!20, 
    text width=3.5em, text badly centered, inner sep=0pt, text height=0.4em, node distance=2cm]
\tikzstyle{block} = [rectangle, draw, fill=blue!20, 
    text width=3em, text centered, minimum height=2em, node distance=2cm]
\tikzstyle{line} = [draw, -latex']
\tikzstyle{cloud} = [draw, ellipse,fill=red!20, node distance=2.5cm,
    minimum height=2em]
    
\begin{tikzpicture}[node distance = 2.2cm, auto]
    % Place nodes

    \node [cloud] (frame) {Frame};
    \node [decision, right of=frame] (motion) {Bewe-\\gung};
    \node [block, right of=motion] (buffer) {Buffer};
    \node [block, right of=buffer] (infer) {Infer};
    \node [decision, right of=infer] (thresshold) {Pred.\\> 0.8};
    \node [cloud, right of=thresshold] (server) {Server};
    
    %\node [block, right of=thresshold] (sender) {Send an Server};

    \path [line] (frame) -- (motion);
    \path [line] (motion) -- node {ja} (buffer);
    \path [line] (buffer) -- (infer);
    \path [line] (infer) -- (thresshold);
    \path [line] (thresshold) -- node {send} (server);

    
    
    
    

    % \node [cloud, left of=init] (expert) {expert};
    % \node [cloud, right of=init] (system) {system};
    % \node [block, below of=init] (identify) {identify candidate models};
    % \node [block, below of=identify] (evaluate) {evaluate candidate models};
    % \node [block, left of=evaluate, node distance=3cm] (update) {update model};
    % \node [decision, below of=evaluate] (decide) {is best candidate better?};
    % \node [block, below of=decide, node distance=3cm] (stop) {stop};
    % % Draw edges
    % \path [line] (init) -- (identify);
    % \path [line] (identify) -- (evaluate);
    % \path [line] (evaluate) -- (decide);
    % \path [line] (decide) -| node [near start] {yes} (update);
    % \path [line] (update) |- (identify);
    % \path [line] (decide) -- node {no}(stop);
    % \path [line,dashed] (expert) -- (init);
    % \path [line,dashed] (system) -- (init);
    % \path [line,dashed] (system) |- (evaluate);
\end{tikzpicture}

    \visible<2->{

    
    \begin{block}{Asynchrone Inferenz}
        \begin{columns}[T]
            \column{0.3\columnwidth}
            \vspace{1cm}
            \begin{itemize}
                \item Asynchrone Inferenzrequests auf mehreren Threads
            \end{itemize}
            \column{0.7\columnwidth}
            \begin{figure}[h]
                \centering
                \def\svgwidth{\columnwidth}
                \input{Bilder/synch_asynch.pdf_tex}
            \end{figure}    
        \end{columns}
    \end{block}
    }
    
\end{frame}

% % Frame 13
% \begin{frame}{Inferenz auf dem NCS2}
%     Implementierung mit OpenVINO Toolkit
%     \begin{block}{Asynchrone Inferenz}

%         \begin{figure}[h]
%             \centering
%             \def\svgwidth{0.7\columnwidth}
%             \input{Bilder/synch_asynch.pdf_tex}
%         \end{figure}

%         \begin{itemize}
%             \item beschleunigte Inferenz durch
%             \begin{itemize}
%                 \item Asynchrone Inferenzrequests auf mehreren Threads
%                 \item Batching
%             \end{itemize} 
%         \end{itemize}
  
%     \end{block}
    
% \end{frame}



% % Frame 14
% \begin{frame}{Anwendung}
%     Ablauf:
%     \begin{enumerate}
%         \item Bewegung: ja/nein
%         \item Zwischenspeichern der Frames
%         \item Inferenz der Frames starten
%         \item Ergebnisse auswerten und an Server schicken
%         \item Benachrichtigung an Nutzer senden
%     \end{enumerate}

% \end{frame}

