w%
%--------------------------------------------------------------------
%
\section[\thesection \ Koordinierte Bewegungen]{Koordinierte
Bewegungen}\label{sec:koordinierteBewegungen}
%
%--------------------------------------------------------------------
%
\subsection[\thesection .\thesubsection \ 
Master--Slave--Anwendungen]{Master--Slave--Anwendungen}\label{sec:masterslave}
%
\begin{frame}{Aufgabenstellung}
%
\begin{block}{Synchronisation der Bewegung mehrerer Achsen:}
%

\vspace{-6pt}

%
\begin{itemize}
  \item Eine Masterachse
  \item Eine oder mehrere Slaveachsen
\end{itemize}
%
\end{block}
%

\vspace{6pt}

\begin{columns}[T]

\column{0.48\textwidth}
\begin{tikzpicture}
%
%--------------------------------------------------------------------
%
% Achse 1
%
\begin{scope}   %Achse1
%
\node[rotate = 90,rectangle,minimum width = 2.5cm,minimum height = 0.8cm,fill =
gray!80!](achse1) at (0,0) {Achse 1};

\begin{scope}

\tikzstyle{every node} = [shape = cylinder,rotate=190,draw,
                          cylinder uses custom fill,cylinder end fill = black!20!,cylinder body fill = black!50!]
%
\node[aspect = 0.85,inner xsep = 0.45cm,inner ysep = 0.4cm,draw] (motor)at
($(achse1.west)+(-0.3,-0.95)$){};
%

\node[aspect = 0.85,inner xsep = 0.15cm,inner ysep = 0.05cm,draw](welle) at
($(motor.east) + (0.27,0.05)$) {};

\end{scope}
%
%Motorleitung
\draw[line width = 2pt]
(achse1.west)--($(achse1.west)+(0,-0.25)$)--($(achse1.west)+(-0.27,-0.3)$)--($(achse1.west)+(-0.27,-0.55)$);
%
\draw[dashed](welle.east)--+(190:1.3)node[left]{$\varphi_1$, $\Omega_1$};
%
\draw[dashed,-latex] ($(welle.east)+(0,-0.5)+(190:0.7)$) arc
(-50:150:0.5);
%
\end{scope}

%
%--------------------------------------------------------------------
%
% Achse 2
%
\begin{scope}[xshift=0.85cm]   %Achse2
%
\node[rotate = 90,rectangle,minimum width = 2.5cm,minimum height = 0.8cm,fill =
gray!80!](achse2) at (0,0) {Achse 2};

\begin{scope}

\tikzstyle{every node} = [shape = cylinder,rotate=190,draw,
                          cylinder uses custom fill,cylinder end fill = black!20!,cylinder body fill = black!50!]

\node[aspect = 0.85,inner xsep = 0.45cm,inner ysep = 0.4cm,draw] (motor) at
($(achse2.west)+(0.34,-1.4)$){};
%

\node[aspect = 0.85,inner xsep = 0.15cm,inner ysep = 0.05cm,draw](welle) at
($(motor.east) + (0.27,0.05)$) {};
%
\end{scope}

\draw[dashed](welle.east)--+(190:3)node[left]{$\varphi_2$, $\Omega_2$};
\draw[dashed,-latex] ($(welle.east)+(0,-0.5)+(190:2.3)$)
arc(-50:150:0.5);

%Motorleitung
\draw[line width = 2pt]
(achse2.west)--($(achse2.west)+(0,-0.5)$)--($(achse2.west)+(0.37,-0.65)$)--($(achse2.west)+(0.37,-1)$);
%
\end{scope}
%
%--------------------------------------------------------------------
%
% Steuerung
%
\node[rotate = 90,rectangle,minimum width = 2.5cm,minimum height = 1.5cm,fill =
gray!80!](sps) at ($(achse1.north) + (-0.8,0)$){Steuerung};
%
\visible<2->{
\draw(sps.north)node[text width = 1.5cm,above left]{Virtueller Master};
\draw(sps.north)node[below left]{$\varphi_{VM}$, $\Omega_{VM}$};
}
%
\end{tikzpicture}

\column{0.47\textwidth}
%
\begin{block}{$\varphi_2$ , $\Omega_2 = f(\varphi_1)$ , $f(\Omega_1)$}
%
\begin{tabular}{ll}
  $\varphi_2 = K \varphi_1$ & Winkelsynchronlauf\\
  $\Omega_2 = K\Omega_1$    & Drehzahlsynchronlauf\\
  $\varphi_2 = f(\varphi_1)$ & Kurvenscheibe
\end{tabular}
%
\end{block}
%
\visible<2->{Masterachse kann auch durch \alert{\em Virtuellen Master} ersetzt
werden}
%
\end{columns}
%
\end{frame}
%
%--------------------------------------------------------------------
%
\begin{frame}{Drehzahlsynchronlauf}
%
Beispiel: Fliegende Säge
%
\begin{itemize}
  \item Master fährt mit konstanter Drehzahl
  %
  \item Slave 
  %
  \begin{enumerate}
    \item Ist im Stillstand in der Position~$X_0$
    \item Beschleunigt auf die Masterdrehzahl
    \item Fährt mit der Masterdrehzahl
    \item Fährt zur Position $X_0$
  \end{enumerate}
  %
  \item Weitere unsynchronisierte Achsen
  %
  \begin{itemize}
    %
    \item Vorschub der Säge von Position $Y_0$ nach $Y_1$
    \item Antrieb der Säge
    %
  \end{itemize}
  %
\end{itemize}
%
\begin{block}{Beispiel Applikationsdaten}
%
\begin{tabular}{llp{0.25cm}ll}
%
Band                &         &&Säge\\[0.25em]
\cline{1-2} \cline{4-5}
%
\\[-0.75em]

%
Geschwindigkeit:     &20m/min  &&Vorschubgeschwindigkeit:&10m/min\\
Breite:              &20cm     &&Beschleunigung X (Slave):&5m/s$^{2}$\\
Länge:               &1m       &&Beschleunigung Y:&6m/s$^{2}$
%
\end{tabular}
%
\end{block}
%
\end{frame}
%
%
%--------------------------------------------------------------------
%
\subsection[\thesection .\thesubsection \ 
Interpolierte Bewegungen]{Interpolierte Bewegungen}\label{sec:interpolation}
%
\begin{frame}{Interpolierte Bewegungen}
%
\begin{block}{Aufgabenstellung}
%
Verfahren einer beliebigen durch die Anwendung vorgegebenen Bahn in der Ebene
/ im Raum
%
\begin{itemize}
  \item Aufteilung der Bewegung in der Ebene / im Raum auf die einzelnen Achsen
  \item Zerlegung der Bahn in Stützstellen zu diskreten Zeitpunkten
  \item Interpolation zwischen den Stützstellen
\end{itemize}
%
\end{block}
%

\vspace{12pt}

Typische Anwendungsfelder
%
\begin{itemize}
  \item Robotik
  \item Werkzeugmaschinen
\end{itemize}
%


\vspace{36pt}


\end{frame}
%
%--------------------------------------------------------------------
%
\begin{frame}{Darstellung einer zweidimensionalen Bahn}
%
\vspace{-3pt}
%
\begin{block}{Implizite Darstellung}
%
\begin{columns}
%
\column{0.2\textwidth}
%
\begin{align}
%
0 = F(X,Y)\nonumber
%
\end{align}
%
\column{0.25\textwidth}
%
Beispiel: Kreis

\vspace{-18pt}

\begin{align}
%
X^2+Y^2 &= 1 \nonumber 
%
\end{align}
%
\column{0.25\textwidth}
%
\begin{tikzpicture}
%
\draw[fill=white,draw] (-0.6,-0.6)rectangle(0.6,0.6);
\draw[thick,black] (0,0) circle(0.3cm);
\draw[-latex,black](-0.5,0)--(0.5,0)node[below]{$\scriptstyle{X}$};
\draw[-latex,black](0,-0.5)--(0,0.5)node[left]{$\scriptstyle{Y}$};
%
\end{tikzpicture}
%
\column{0.25\textwidth}
%
%
\end{columns}
%
\end{block}

\vspace{-6pt}

\begin{block}{Explizite Darstellung}
%
\begin{columns}
%
\column{0.2\textwidth}
%
\begin{align}
%
X = F_1(Y)\nonumber\\
Y = F_2(X)\nonumber
%
\end{align}
%
\column{0.25\textwidth}
%
Beispiel: Kreis

\vspace{-18pt}

\begin{align}
%
X &= \sqrt{1-Y^2} \nonumber\\
Y &= \sqrt{1-X^2} \nonumber 
%
\end{align}
%
\column{0.25\textwidth}
%
\begin{tikzpicture}
%
\draw[fill=white,draw] (-0.6,-0.6)rectangle(0.6,0.6);
\draw[thick,black] (0,0) circle(0.3cm);
\draw[-latex,black](-0.5,0)--(0.5,0)node[below]{$\scriptstyle{X}$};
\draw[-latex,black](0,-0.5)--(0,0.5)node[left]{$\scriptstyle{Y}$};
%
\end{tikzpicture}
%
\column{0.25\textwidth}
%

%
\end{columns}
%
\end{block}
%

\vspace{-6pt}

%
\begin{block}{Parametrische Darstellung}
%
\begin{columns}
%
\column{0.2\textwidth}
%
\begin{align}
%
X = F_1(u)\nonumber\\
Z = F_2(u)\nonumber
%
\end{align}
%
\column{0.25\textwidth}
%
Beispiel: Kreis
%

\vspace{-18pt}

\begin{align}
%
X &= \sin{u} \nonumber\\
Z &= \cos{u} \nonumber 
%
\end{align}
%
\column{0.15\textwidth}
%
\begin{tikzpicture}
%
\draw[fill=white,draw] (-0.6,-0.6)rectangle(0.6,0.6);
\draw[thick,black] (0,0) circle(0.3cm);
\draw[-latex,black](-0.5,0)--(0.5,0)node[below]{$\scriptstyle{X}$};
\draw[-latex,black](0,-0.5)--(0,0.5)node[left]{$\scriptstyle{Y}$};
%
\end{tikzpicture}
%
\column{0.2\textwidth}
%
Beispiel: Kurve

\vspace{-18pt}

\begin{align}
%
X &= u\sin{u} \nonumber\\
Y &= u\cos{u} \nonumber 
%
\end{align}
%
\column{0.1\textwidth}
%
\begin{tikzpicture}
%
\draw[fill=white,draw] (-0.3,-0.6)rectangle(0.6,0.6);
\draw[-latex,black](-0.2,0)--(0.5,0)node[below]{$\scriptstyle{X}$};
\draw[-latex,black](0,-0.5)--(0,0.5)node[left]{$\scriptstyle{Y}$};
%
\draw[thick,black,scale=0.15,domain=-2.9:2.9,smooth,variable=\t]
plot ({\t*sin(\t r)},{\t*cos(\t r)});
%
\end{tikzpicture}
%
\end{columns}
%
\end{block}
%
\end{frame}
%
%--------------------------------------------------------------------
%
\begin{frame}{Satzübergänge}
%
Übergang zwischen den Teilstücken des interpolierten Bewegungsprofils
%
\begin{enumerate}
  \item Standardverfahren
\end{enumerate}
%
\begin{columns}[T]
%
\column{0.1\textwidth}
%
\column{0.25\textwidth}
%
\begin{itemize}
  \item Übergang mit {\em Genauhalt}
\end{itemize}
%
\column{0.6\textwidth}
\begin{tikzpicture}
%
\coordinate (start) at (0,1);
\coordinate (ecke) at (1,1);
\coordinate (semitte) at ($(start)!0.65!(ecke)$);
\coordinate (ende) at (1,0);
\coordinate (eemitte) at ($(ecke)!0.65!(ende)$);
%
\draw[-latex,thick](start)--(semitte);
\draw[-latex,thick](ecke)--(eemitte);
\draw[thick](start)--(ecke)--(ende);
%
\begin{scope}[xshift = 3cm]
%
\draw[-latex](-0.1,0)--(3.5,0);
\draw[-latex](0,-0.1)--(0,1)node[left]{$V_X,V_Y$};
%
\draw[thick,blue](0,0.6)--(1.25,0.6)--(1.5,0);
\draw[thick,dashed,red](1.5,0)--(1.75,0.6)--(3,0.6);
%
\end{scope}
%
\end{tikzpicture}
%
\end{columns}
%
\vspace{12pt}
%
\begin{columns}[T]
%
\column{0.1\textwidth}
%
\column{0.25\textwidth}
%
\begin{itemize}
  \item Übergang mit {\em Überschleifen}
\end{itemize}
%
\column{0.6\textwidth}
\input{./Bilder/satzuebergangueberschleifen.tex}
%
\end{columns}
%
\vspace{12pt}
%
\begin{columns}[T]
%
\column{0.1\textwidth}
%
\column{0.25\textwidth}
%
\begin{itemize}
  \item Übergang mit {\em Grenzgeschwindigkeit}
\end{itemize}
%
\column{0.6\textwidth}
\input{./Bilder/satzuebergangvmin.tex}
%
\end{columns}
%

\vspace{6pt}

Satzübergänge ohne Genauhalt
%
\begin{itemize}
  \item[$\Rightarrow$] Eckenverrundung aufgrund der Maschinendynamik / der
  Schleppfehler
\end{itemize} 
%
\end{frame}
%
%
%--------------------------------------------------------------------
%
\begin{frame}{Satzübergänge}
%
\begin{enumerate}
  \setcounter{enumi}{0}
  \item Standardverfahren (Fortsetzung)
\end{enumerate}
%
\vspace{6pt}
%
\begin{columns}
%
\column{0.1\textwidth}
%
\column{0.85\textwidth}
%

Beispiel: Geradeninterpolation zwischen drei Punkten

\vspace{6pt}

%
\begin{tikzpicture}
%
%Positionen/Kontur
%
\begin{scope}[xscale = 0.15,yscale = 0.15]
%
\draw[-latex] (-1,0)node[left]{$Y_0$}--(25,0)node[below]{$X$};
\draw[-latex] (0,-1)node[below]{$X_0$}--(0,20)node[left]{$Y$};
%
\draw[thick](0,0)--(7,7)--(20,14.5);
%
\draw[dashed](0,7)node[left]{$Y_1$}--(7,7)--(7,0)node[below]{$X_1$};
\draw[dashed](0,14.5)node[left]{$Y_2$}--(20,14.5)--(20,0)node[below]{$X_2$};
%
\end{scope}
%
\begin{scope}[xshift = 4.5cm,xscale = 0.01,yscale = 0.01]
%
\draw[-latex] (-10,0)--(460,0)node[below]{$t$};
\draw[-latex] (0,-10)--(0,125)node[left]{$V_X$};
%
\draw[thick](0,0)--(84,84)--(168,0)node[below]{$t_1$}--(294,103)--(420,0)node[below]{$t_2$};
%
\draw[dashed](84,0)--(84,230);
\draw[dashed](168,0)--(168,150);
\draw[dashed](294,0)--(294,200);
\draw[dashed](420,0)--(420,150);
%
\end{scope}
%
\begin{scope}[xshift = 4.5cm,yshift = 1.5cm,xscale = 0.01,yscale = 0.01]
%
\draw[-latex] (-10,0)--(460,0)node[below]{$t$};
\draw[-latex] (0,-10)--(0,125)node[left]{$V_Y$};
%
\draw[thick](0,0)--(84,84)--(168,0)--(294,59.5)--(420,0);
%
\end{scope}
%
\end{tikzpicture}
%

\vspace{6pt}
%

Die gesamte Bahn setzt sich aus einzelnen $R$--$R$--Segmenten zusammen

%
\end{columns}
%
\vspace{12pt}
%
\end{frame}
%
%--------------------------------------------------------------------
%
\begin{frame}{Satzübergänge}
%
Übergang zwischen den Teilstücken des interpolierten Bewegungsprofils
%
\begin{enumerate}
  \setcounter{enumi}{1}
  \item Übergang mit {\em Look--Ahead}
\end{enumerate}
%
\begin{columns}
%
\column{0.1\textwidth}
%
\column{0.85\textwidth}
%

Forsetzung Beispiel: Geradeninterpolation zwischen drei Punkten

\vspace{6pt}

%
\begin{tikzpicture}
%
%Positionen/Kontur
%
\begin{scope}[xscale = 0.15,yscale = 0.15]
%
\draw[-latex] (-1,0)node[left]{$Y_0$}--(25,0)node[below]{$X$};
\draw[-latex] (0,-1)node[below]{$X_0$}--(0,20)node[left]{$Y$};
%
\draw[thick](0,0)--(7,7)--(20,14.5);
%
\draw[dashed](0,7)node[left]{$Y_1$}--(7,7)--(7,0)node[below]{$X_1$};
\draw[dashed](0,14.5)node[left]{$Y_2$}--(20,14.5)--(20,0)node[below]{$X_2$};
%
\end{scope}
%
%Geschwindigkeiten x
%
\begin{scope}[xshift = 4.5cm,xscale = 0.01,yscale = 0.01]
%
\draw[-latex] (-10,0)--(460,0)node[below]{$t$};
\draw[-latex] (0,-10)--(0,125)node[left]{$V_X$};
%
\draw[thick](0,0)--(84,84)--(126,84)--(140,103)--(192,103)--(295,0)node[below]{$t_2$};
%
\draw[dashed](84,0)--(84,230);
\draw[dashed](126,0)node[below]{$t_1$}--(126,230);
\draw[dashed](140,0)--(140,200);
\draw[dashed](192,0)--(192,200);
\draw[dashed](295,0)--(295,150);

%
\end{scope}
%
%Geschwindigkeiten y
%
\begin{scope}[xshift = 4.5cm,yshift = 1.5cm,xscale = 0.01,yscale = 0.01]
%
\draw[-latex] (-10,0)--(460,0)node[below]{$t$};
\draw[-latex] (0,-10)--(0,125)node[left]{$V_Y$};
%
\draw[thick](0,0)--(84,84)--(126,84)--(140,59.5)--(192,59.5)--(295,0);
%
\end{scope}
%
\end{tikzpicture}
%

\vspace{6pt}
%

\begin{itemize}
  \item Keine R--R Bewegungen mehr
  \item Geringere Geschwindigkeitsänderungen
\end{itemize}

%
\end{columns}
%
\vspace{12pt}
%
\end{frame}
%