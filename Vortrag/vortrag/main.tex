\documentclass[pdftex,10pt]{beamer}

\useoutertheme{infolines}

\usepackage[ngerman]{babel}
\usepackage[utf8]{inputenc}
\usepackage{times}
\usepackage[T1]{fontenc}
\usepackage{graphicx}
\usepackage[right]{eurosym}
\usepackage{longtable}
\usepackage{bibgerm}

\usepackage{tikz}
\usetikzlibrary{positioning}
\usetikzlibrary{calc}
\usetikzlibrary{fit}
\usetikzlibrary{shapes.geometric}
\usetikzlibrary{shapes.arrows}
\usetikzlibrary{decorations}
\usetikzlibrary{decorations.pathmorphing}
\usetikzlibrary{decorations.text}
\usetikzlibrary{backgrounds}
\usetikzlibrary{intersections}
\usetikzlibrary{plotmarks}
\usetikzlibrary{arrows.meta}

\usepackage{pgfplots}
\pgfplotsset{compat=1.10}

\usepackage[european]{circuitikz}
\ctikzset {bipoles/length=1cm}

\setlength{\arrayrulewidth}{0.75pt}

\title[Autonomes Bilderkennungsystem, Neuronale Netze] {Entwicklung eines autonomen Systems zur Bilderkennung mithilfe Neuronaler Netze auf dedizierter Hardware}

\subtitle{\vspace{0.5cm}Kolloqiuim - Bachelorarbeit}

\author[Manuel Barky]{Manuel Barkey}

  
\date[Reutlingen, 29.01.2020] 
{Reutlingen, 29.01.2020}

\subject{Mechatronik}


\pgfdeclareimage[height=0.49cm]{logotitel}{./Bilder/Silhouette_HSRT_045K.jpg}
\pgfdeclareimage[height=0.235cm]{logofolie}{./Bilder/hsrt_silhouette_folie.png}
\logo{\pgfuseimage{logotitel}}


% Falls Aufzählungen immer schrittweise gezeigt werden sollen, kann
% folgendes Kommando benutzt werden:

%\beamerdefaultoverlayspecification{<+->}

\begin{document}

\begin{frame}
  \titlepage
\end{frame}

\logo{\pgfuseimage{logofolie}}

%Einleitung soll direkt nach der Titelfolie erfolgen, ohne zusätzliche
%Inhalts- und Inhaltsverlaufsfolien
%
%
%Nach der Einleitung wird vor jedem Abschnitt das Inhaltverzeichnis eingeblendet

\section*{Motivation}\label{sec:motivation}


\begin{frame}{Motivation}

    \begin{columns}[T]
        \column{0.5\columnwidth}
        Spalte 1
        \begin{itemize}
            \item autonomes überwachungssystem, (wild) tiere, tag/nacht geeignet
            \item rasperry pi + infrarotfähiges camera modul
            \item nur mittteilen bei relevanten erkennungen -> NN
            \item on the edge -> spezielle hardware: NCS2
        \end{itemize}
        \column{0.5\columnwidth}
        Spalte 2\\
        irgendwelche bilder
        
    \end{columns}
\end{frame}


\AtBeginSection[]
{
  \begin{frame}<beamer>{Gliederung}
    \tableofcontents[currentsection,hideothersubsections]
  \end{frame}
} 

\begin{frame}{Gliederung}
  \tableofcontents[hideallsubsections]
\end{frame}
%
\section[\thesection \  Künstliche Neuronale Netze]{Künstliche Neuronale Netze}\label{sec:nn}

\begin{frame}{Neuronale Netze}

    \begin{itemize}
        \item was: NN lernt in gr Datenmenge Zusammenhänge und kann diese generalisieren so das es sie auch für neue daten anwenden kann
        \item wie: input daten mit zugehörigen outputs (hier gelabelte bilder) in Modell, dieses lernt iterativ die zusammenhänge
    \end{itemize}

    \vspace{1cm}

    \begin{center}
        \begin{columns}[T]
            \column{0.3\textwidth}
            \begin{tikzpicture}[scale=0.5]
    \node at (0,1.5) {\scriptsize data};
    \node at (0,0.5) {\scriptsize regeln};
    \draw (0.6,1.5) -- (1.5,1.5);
    \draw (0.6,0.5) -- (1.5,0.5);
    \draw (1.5,0) rectangle (5.5,2);
    \node at (3.5,1) {\scriptsize classic};
    \draw (5.5, 1) -- (6.3, 1);
    \node at (7, 1) {\scriptsize output};
\end{tikzpicture}
            \column{0.3\textwidth}
            \begin{tikzpicture}[scale=0.5]
    \node at (0,1.5) {\scriptsize data};
    \node at (0,0.5) {\scriptsize output};
    \draw (0.6,1.5) -- (1.5,1.5);
    \draw (0.6,0.5) -- (1.5,0.5);
    \draw (1.5,0) rectangle (5.5,2);
    \node at (3.5,1) {\scriptsize ml Programm};
    \draw (5.5, 1) -- (6.3, 1);
    \node at (7, 1) {\scriptsize regeln};
\end{tikzpicture}
        \end{columns}
        
    \end{center}
    
\end{frame}

\section[\thesection \  Hardware]{Hardware}\label{sec:hardware}

% Frame 4
\begin{frame}{Intel Neural Compute Stick 2}
    \begin{columns}[T]
        \column{0.6\textwidth}
        Beschleuniger für die Inferenz von Deep Learning Algorithmen
        \vspace{0.3cm}
        \begin{itemize}
            \item geeignet für \textbf{Edge Anwendungen} wie:
            \begin{itemize}
                \item Überwachungskameras, Drohnen, \dots
            \end{itemize}
            \end{itemize}
            \begin{itemize}
                
            
            \item Prozessor: \textbf{Intel Movidius Myriad X VPU}
            \begin{itemize}
                \item effizient bei NN-spezifischen Rechenopereationen
            \end{itemize}
            
        \end{itemize}
        \column{0.4\textwidth}
        \vspace{1cm}
        \includegraphics[width=0.8\textwidth]{Bilder/ncs2.jpg}
    \end{columns}
    \vspace{0.3cm}

    \visible<2->{

    \begin{columns}[T]
        \column{0.4\textwidth}

        \begin{block}{OpenVINO Toolkit}
    
            Inferenz auf Intel-Hardware
            \begin{itemize}
                \item Eigenes Dateiformat für Deep Learning Model
                \item Unterstütze Frameworks:
                \begin{itemize}
                    \item  Tensorflow, Caffe
                \end{itemize}
            \end{itemize}
        \end{block}
            \column{0.6\textwidth}
            \includegraphics[width=\textwidth]{Bilder/open_vino_workflow_steps.png}
        \end{columns}

    }
        
\end{frame}
\section[\thesection \  Training des Modells]{Training des Modells}\label{sec:training}
%
%--------------------------------------------------------------------
%
\subsection[\thesection .\thesubsection \ 
Sammeln und aufbereiten der Daten]{Sammeln und aufbereiten der Daten}\label{subsec:collect_data}
%
\begin{frame}{Datenset}

\end{frame}

\begin{frame}{Augmentierung}
    
\end{frame}

\subsection[\thesection .\thesubsection \ 
Auswahl und Training des Modells]{Auswahl und Training des Modells}\label{subsec:train_model}

\begin{frame}{CNNs}

\end{frame}

\begin{frame}{Objekterkennung}
    
\end{frame}

\begin{frame}{Tensorflow ObjDet Api}
    
\end{frame}

\subsection[\thesection .\thesubsection \ 
Evaluierung des Trainings]{Evaluierung des Trainings}\label{subsec:eval}

\begin{frame}{Loss und map}
    auf validation set

\end{frame}

\begin{frame}{inferenz}

    auf test set
    
\end{frame}

%##########################  CHAPER 6: APPLICATION  #######################

\chapter{Entwicklung der Anwendung}\label{kap:application}


Dieses Kapitel beschreibt die Realisierung der
Anwendung, als autonomes Kamerasystem zur
Wildtiererkennung.

Zunächst werden dabei die verwendeten Hardwarekomponenten 
erläutert.

Im zweiten Abschnitt wird die Implementierung der 
Inferenz, für eines der trainierten Modelle, 
sowie einer geeigneten Kommunikationsmöglichkeit 
zur Übertragung der Daten beschrieben.


%-------------------------  SECTION 1: AUFBAU  ------------------------
\section{Hardware}\label{sec:aufbau}


Der Aufbau der Anwendung besteht aus einem, in Abbildung 
\ref{fig:raspberrypi} dargestellten Raspberry Pi 4,
auf dem der Programmcode ausgeführt wird,
sowie dem Neural Compute Stick 2 für die Inferenz,
welcher über eine USB Schnittstelle
mit dem Raspberry Pi verbunden wird.

Zur Aufnahme der Bilder wurde das in 
Abbildung \ref{fig:rpicam} dargestellte 
Raspberry Pi Kamera Modul,
mit einem 5MP OV5647 Sensor der Marke Longrunner
verwendet.
Dieses ermöglicht, durch mechanisches zu und abschalten
eines Infrarot Filters vor die Linse, zwischen Tag- und
Nachtsicht zu wechseln.
Der dafür verwendete Magnetschalter wird automatisch 
über einen Helligkeitsensor getriggert.
Im Infrarotmodus befindet sich der Filter nicht 
vor der Linse, sodass neben den elektromagnetischen 
Wellen des Sichtbaren Lichts, auch die des 
langwelligeren des Infrarot Spektrums (850nm) 
auf die Linse treffen und verarbeitet werden können.

Zudem verfügt die Kamera über zwei Infrarot LEDs, 
sodass auch Aufnahmen, in bis zu 3m Entfernung,
in völliger Dunkelheit gemacht werden können.
Diese haben den Vorteil gegenüber normalen LEDs, 
dass die Tiere von keiner Sichtbaren Lichtquelle 
gestört oder verscheucht werden.

Verbunden wird das Kamera Modul über die CSI 
(Camera Serial Interface) 
Schnittstelle des Raspberry Pi's.

\vspace{1cm}
%https://www.amazon.de/gp/product/B07R4JH2ZV/ref=ppx_yo_dt_b_asin_title_o01_s00?ie=UTF8&psc=1
\begin{minipage}{0.55\textwidth}
    \centering
    \includegraphics[width=0.8\textwidth]
    {./Bilder/raspberrypi_4.png}
    \captionof{figure}{Raspberry Pi 4}
    \label{fig:raspberrypi}
\end{minipage}
\begin{minipage}{0.45\textwidth}
    \centering
    \includegraphics[width=0.8\textwidth]
    {longrunner.jpg}
    \captionof{figure}{Longruner Kamera Modul}
    \label{fig:rpicam}
\end{minipage}
\vspace{1cm}


Desweiteren wurde für eine mobile Internetverbindung 
der \textit{Huawei E3531 SurfStick} und zu Stromversorgnung
eine Powerbank verwendet.



% https://www.amazon.de/gp/product/B00HSZEY34/ref=ppx_yo_dt_b_asin_title_o00_s00?ie=UTF8&psc=1


\section{Software}

Die Implementierung der Applikation für den Raspberry Pi
wurde in Python vorgenommen. 
Dabei sind die Funktionalitäten zur Objekterkennung in
dem Script \textit{detection.py} 
und die, zur Herstellung einer Verbindung
und Senden der Daten, in dem 
\textit{connection.py} Script definiert.


Der Kamera Inputstream ist in einem \textit{main.py} Script 
implementiert, von dem aus auch die in Abbildung 
\ref{fig:class_diagram} 
dargestellten Klassen, welche in \textit{detection.py}
und \textit{connection.py} enthalten sind,
verwendet werden.


\vspace{1cm}
\begin{minipage}{0.75\textwidth}
    \centering
    \underline{detection.py}
\end{minipage}
\begin{minipage}{0.25\textwidth}
    \centering
    \underline{connection.py}
\end{minipage}
\begin{figure}[H]
    \centering
    \newcommand\restoreuscatcode{\catcode`\_=8 }
\tikzset{every picture/.prefix style={execute at begin picture=\restoreuscatcode}}


\begin{tikzpicture}   
    
            \umlclass[x=-0.5, y=0.2]{Motion}{ 
              statick\text{\_}background : np.array
              }{ 
              + detect\text{\_}motion(frame) : bool: \\
              + reset\text{\_}background() : void
            }
        
            \umlclass[y=-4.2]{InferenceModel}{ 
                plugin : ie\text{\_}api.IEPlugin\\
                string : device
                }{ 
                + init(device)\\  
                + create\text{\_}exec\text{\_}infer\text{\_}model(.xml, .bin, NReq)\\: ExecInferModel
              }
        
            \umlclass[y=-1, x=5.6]{ExecInferModel}{ 
                exec\text{\_}net : ie\text{\_}api.ExecutableNetwork\\
                labels : list\\
                input\text{\_}blob : \textit{input\text{\_}shape}\\
                output\text{\_}blob : \textit{output\text{\_}shape}\\
                current\text{\_}frame : dict()\\
                detected\text{\_}objects : dict()

                }{ 
                + init(exec\text{\_}net, labels, input\text{\_}blob)\\
                + infer\text{\_}frames() : \textit{status} \\
                - save(class\text{\_}id) : void
                }
        
    
        \umlclass[x=11.5, y=-2]{Connection}{ 
            loin\text{\_}data : string
            }{ 
            + login(device\text\_name) : bool \\
            + connect() : (server, port)\\
            + send(server, port, file) : bool\\
            + disconnect() : bool\\
            + send\text{\_}email(addr, msg) : bool
          }

    
    \end{tikzpicture}
            
    \caption{Klassendiagramm der Anwendung}
    \label{fig:class_diagram}
\end{figure}
\vspace{1cm}


Die Klasse \textit{Motion} dient zur Erkennung von Bewegungen 
im Kamera Input Stream, \textit{InferenceModel} und
\textit{ExecInferModel} realisieren die Inferenz 
eines trainierten Modells und die Klasse 
\textit{Connection} dient dem Aufbau einer
Verbindung zu einem anderen Gerät, sowie dem Senden
der erkannten Bilder darüber.


Durch geeigneten Implementierung des Applikationsablaufes,
sollte eine Möglichkeit gefunden werden, trotz 
der langsamen Inferenzzeit, mit dem Faster R-CNN
alle relevanten Frames, also die, in denen Tiere zu vermuten
sind, inferieren zu können.
Dafür wurde die Annahme gemacht, dass zur Laufzeit der 
Anwendung, nicht durchgehend inferiert werden muss,
sich also Zeitweise keine Tiere und damit auch keine 
Bewegung vor der Kamera befinden.

Um Bewegungen feststellen zu können, 
wurde, mithilfe der Library \textit{OpenCV}
ein Bewegungsmelder implementiert.
Dieser speichert zu Begin des Kamera Streams ein Referenz
Frame ab, mit dem alle weiteren Frames verglichen werden.
Beträgt der Abstand, der einzelnen Pixelwerte im 
Graustufenbereich mehr, als ein bestimmter 
Threshhold, wird dies als Bewegung gewertet.
Indem nun die Frames, welche der Kamera Stream permanent 
liefert, zunächst auf Bewegung überprüft werden, 
lässt sich unnötiges inferieren vermeiden,
was Zeit und und Energie kostet.
Frames, die Bewegung enthalten, und aufgrund der langsamen 
Inferenzzeit des Faster R-CNN nicht sofort inferiert 
werden können, werden in einem Buffer zwischen 
gespeichert und in Phasen, zu denen keine Bewegung stattfindet, 
inferiert.

Dafür musste der in Abschnitt \ref{sec:infertime} beschriebene
asynchrone Inferenzablauf dahingehend angepasst werden,
dass kein blockierendes warten auf 
ein Inferenzergebnis stattfindet,
wodurch die Inferenz komplet zeitasynchron zu 
den Inputframes abläuft.
Der Gesamtablauf der Applikation ist in Abbildung 
\ref{fig:flowchart_appl} 
schematisch als Flussdiagram dargestellt.

\vspace{1cm}
\begin{figure}[H]
    \centering
    \tikzset{
    desicion/.style={
        diamond,
        draw,
        text width=4em,
        text badly centered,
        inner sep=0pt
    },
    block/.style={
        rectangle,
        draw,
        text width=10em,
        text centered,
        rounded corners
    },
    arrow/.style={
        draw,
        >=latex,
        ->
    }
}


\begin{tikzpicture}
    \node (A) [desicion] {entschei\\dung};
    \node (B) [block, below of=A, node distance=3cm, text width=5em] {bock};
    \node (C) [block, right of=A, node distance=0.5\textwidth] {noch ein\\bock};


    \draw[arrow] (A) --  node [left, fill=white!30] {yes} (B);
    \draw[arrow] (A) -- node [below, near end] {crap} (C); 
    \draw[arrow] (B) -| node [near start, fill=white] {yes} (C);

\end{tikzpicture}
    
    \caption{Schematischer Ablauf des Applikationscode}
    \label{fig:flowchart_appl}
\end{figure}
\vspace{1cm}


Wird in inferierten Frames mehrfach nichts 
erkannt, wird das Referenz Frame des 
Bewegungsmelders durch ein aktuelles Frame ersetzt.

Erkannte Objekt werden in einer
Datenstruktur (Pytho Dictionary) zusammen 
mit Klassenname (cls), Wahrscheinlichkeit(p)
Anzahl an Erkennungen (N) sowie die Bounding 
Box Koordinaten (Roi, (Region of Interest)) 
abgespeichert.

Nach einer bestimmte Anzahl 
an Erkennungen des selben Objekts, wird dieses 
als lokale Bilddatei abgespeichert und ein 
Send-Requst an das Main Script zurückgegeben.

Dieses prüft dann ob eine Verbindung zu einem 
anderen Gerät besteht, stellt diese gegebenenfalls her,
und sendet die lokal abgespeicherten Bilder.

Um nicht permanent die Verbindung zu einem Pc aufrecht erhalten 
zu müssen, was das Datenvolumen des mobilen Internets
schneller aufbrauchen würde, wird diese nach einer 
bestimmte zeit ohne Bewegung getrennt.

Im folgenden werden die Funktionsweise der 
Inferenz sowie der Verbindungsaufbau 
genauer erklärt.


\subsection*{Inferenz}

Der im Abschnitt \ref{sec:infertime} beschriebene asynchronen
Inferenzablauf wurde dahingehend angepasst, dass eine beliebige
Anzahl an Inferenz Requests verwendet werden kann 
und dass das Warten auf ein Inferenz Ergebnis
nicht mehr blockierend ist.
Dafür wurde der Timeout in der Wait-Funktion auf 
$0ms$ gesetzt.
In Algorithmus \ref{code:infer_async_neu} ist 
der Inferenzablauf als Pseudocode dargestellt.

\begin{algorithm}[H]
    \caption{Asynchrone Inferenz, ohne Blockierung}
    \label{code:infer_async_neu}
    \begin{algorithmic}
    \WHILE{\TRUE}
    \STATE capture FRAMES
        \FOR{all InferRequests}
            %\STATE Status $\leftarrow$ \textbf{wait} for 
            % InferRequest
            \IF {\textbf{wait} for InferRequest \textbf{is} 0}
                \STATE Result $\leftarrow$ InferRequest.output
            \ENDIF
            \IF {Buffer \textbf{not} empty}
                \STATE preprocess InferRequest
                \STATE \textbf{start} InferRequest
            \ENDIF
            \IF{Result not NULL}
                \STATE process Result
            \ENDIF
        \ENDFOR
    \ENDWHILE
    \end{algorithmic}
\end{algorithm}    





\subsection*{Connection}

Um die Bilder mit erkannten Tieren an ein anderes Gerät 
z.B. einen Pc senden zu können, musste eine Verbindung
hergestellt werden, die auch über verschiedene Netzwerke 
hinweg funktioniert.

Um unabhängig von Router Konfigurationen und Firewall 
Einstellungen zu sein, wurde mithilfe des
Dienstes \textit{remot3.it} \cite{remoteit}
eine Cloudbasierte Remote Verbindung hergestellt.

Mit dieser war es möglich eine Remote Proxy SSH Verbindung, 
über das Internet zu einem anderen Gerät herzustellen.

\begin{figure}[H]
    \centering
    \def\svgwidth{0.7\textwidth}
    \input{Bilder/diagram-connect.pdf_tex}
    \caption{Prinzip Proxy Verbindung}
    \label{fig:remoteit}
\end{figure}

Da die Daten vom Raspberry Pi aus automatisiert gesendet 
werden sollen, wurde der Pc, an den 
sie Daten gesendet werden, als Remote Gerät implementiert.

Gesendet wurden die Daten über das \textit{Secure Copy Protocol},
welches das hergestellte \textit{Secure Shell Protocol (SSH)}
verwendet.
Dieses lässt sich über folgendes Kommando,
welches im \textit{connection.py}
Script ausgeführt wird, bedienen:
\begin{itemize}
    \item[\texttt{\$}] \texttt{scp -P port file.jpg 
    user@proxyadresse /zielpfad/file.jpg}
\end{itemize}

Server und Port werden dabei von remote.it
generiert, \textit{file.jpg} ist das zu sendende Bild und
\textit{user} der Nutzername des Geräts,
an welches gesendet wird.
Um das Einloggen sowie den Verbindungs Auf- und Abbau 
über remote.it zu einem Gerät automatisieren zu können,
bietet remote.it eine API mit der über Post- und Get Requests
die Befehle dafür programmatisch aufgerufen werden können.

Um den Nutzer bei einer Erkennung automatisch 
per Email zu benachrichtigen, wurde 
eine Funktion implementiert, welche das 
\textit{Smart Mail Transfer Protpkol (SMTP)} 
verwendet.

\chapter{Zusammenfassung}\label{kap:zusammenfassungausblick}


Ziel der Arbeit war es, ein autonomes
Kamerasystem zur Wildtierkennung
mithilfe Neuronaler Netze zu entwickeln.
Dafür wurden vortrainierte, \Gls{cnn}-basierte, 
Deep-Learning-Modelle zur Objekterkennung
auf einen geeigneten Datensatz trainiert.
Dadurch sollte es möglich sein, nicht 
nur die Anwesenheit eines Tieres 
zu erkennen, sondern auch eine 
Klassifizierung der Tierart 
vorzunehmen. Auf diese Weise kann das System 
gezielter für eine bestimmte
Anwendung eingesetzt werden.
Zur Realisierung wurde neben einem \textit{Raspberry
Pi} sowie einer nachtsichtgeeigneten 
Kamera der \textit{Neural Compute Stick 2}
von \textit{Intel} verwendet, um die 
Verarbeitung der Daten auf dem 
Gerät ausführen zu können.
\vspace{0.5cm}


Für das Training wurde ein Datensatz, 
bestehend aus 9 Wildtierklassen verwendet,
welcher aus \textit{Open Images} heruntergeladen werden konnte.
Anschließend wurden die Daten 
für das Training aufbereitet und
zur Evaluierung in verschiedene 
Sets aufgeteilt.
Das Training wurde dann mithilfe des 
\Glspl{framework} \textit{TensorFlow} durchgeführt,
wobei die Modelle Single Shot Detector (SSD) und
Faster R-CNN mit verschiedenen
Basis-\Glspl{cnn} und Parametereinstellungen
verwendet wurden.
Durch anschließende Evaluierung konnte 
festgestellt werden, welches Modell sich,
bezogen auf Genauigkeit und Geschwindigkeit,
am besten für die Anwendung eignet.
Der letzte Schritt war die Inferenz zusammen 
mit dem Anwendungscode für den \textit{Raspberry Pi} 
zu implementieren, wofür mit \textit{OpenVino} gearbeitet 
wurde.
\vspace{0.5cm}

Die Evaluierung der trainierten Modelle zeigte,
dass eine erhöhte Genauigkeit mit einer 
langsameren Inferenzzeit einhergeht.
Durch umfangreiche Testläufe mit 
variierenden Parametern konnten die 
optimalen Konfigurationen für die Anwendung 
erforscht und somit die Ergebnisse verbessert 
werden.

Daraus resultierend wurde für die Anwendung das 
Faster R-CNN Modell mit InceptionV2 
als Basis-\Gls{cnn} gewählt. Dieses erreichte 
durch ein Training von 500k Iterationen 
auf den \textit{Open Images} Datensatz 
einen mAP-Wert von 0.7 und einen Loss-Wert von 0.74.
Der Datensatz wurde durch geometrische und 
pixelwertbezogene Veränderungen der Bilder 
augmentiert, wodurch für jede Klasse 3000 
Bilder für das Training vorhanden waren.

Durch einen asynchronen Inferenzablauf 
mit drei Inferenz-Requests konnte die 
Inferenzzeit für das Faster R-CNN 
von 0,63 \Gls{fps} auf 0,75 \Gls{fps} 
erhöht werden. 
Indem ein Bewegungsmelder sowie ein 
Zwischenspeichern der Frames im 
Anwendungsablauf implementiert wurde, ließ sich 
die Performance weiter verbessern.

\section[\thesection \  Positionieren von Einzelachsen]{Positionieren von
Einzelachsen}\label{sec:positioneinzelachsen}
%
%--------------------------------------------------------------------
%
\subsection[\thesection .\thesubsection \ 
Bewegungsprofil]{Bewegungsprofil}\label{sec:bewegungsprofil}
%
\begin{frame}{Einfaches Streckenmodell}
%
Einfaches Streckenmodell für die prinzipiellen Zusammenhänge
%
\begin{itemize}
  \item Kein Getriebe
  \item Starre Welle
  \item $J_G=J_M+J_L$
\end{itemize}
%
\begin{center}
%
\begin{tikzpicture}
%
\draw(0,0)node[rectangle,minimum width = 1.8cm,minimum height =
1.2cm,draw](m2n){};
\draw(m2n.south west)--(m2n.north east);
\draw(m2n.north west)node[above]{$\dfrac{1}{J_G}$};
%
\draw($(m2n.east)+(1,0)$)node[rectangle,minimum width = 1.8cm,minimum height =
1.2cm,draw,anchor = west](n2x){};
\draw(n2x.south west)--(n2x.north east);
%
\draw($(m2n.west)+(-1,0)$)node[circle,minimum width = 0.2cm,draw](deltam){};
%
%---------------
%
\draw[-latex](m2n.east)--(n2x.west)node[pos = 0.5,above]{$\Omega$};
\draw[-latex](deltam.east)--(m2n.west);
%
\coordinate (mm) at ($(deltam.west)+(-1,0)$);
\coordinate (ml) at ($(deltam.north)+(0,1)$);
\coordinate (xist) at ($(n2x.east)+(1,0)$);
%
\draw[-latex] (mm)node[above right]{$M_M$}--(deltam.west);
\draw[-latex] (ml)node[below left]{$M_L$}--(deltam.north)node[above left]{$-$};
\draw[-latex] (n2x.east)--(xist)node[above left]{$\varphi$};
%
\end{tikzpicture}
%
\end{center}
%

\vspace{12pt}

Für die folgenden Überlegungen gilt~$M_L=0$.

\vspace{6pt}

Bezeichung: $\ddot \varphi =\dot \Omega = \alpha$

\vfill

%
\end{frame}
%
%--------------------------------------------------------------------
%
\begin{frame}{Drehmomentvorgabe}
%
\begin{tikzpicture}
%
%Bewegungsprofil fuer J = 10e-4kgm^2
%
\draw[dashed,ystep=0.8] (0,0) grid (4,2.4);
\draw[line width = 0.5pt] (0,0) rectangle (4,2.4);
%
\draw(-0.8,2)node[left,rotate = 90]{$M_M/Nm$};
\foreach \y/\ytext in {0/0,0.8/0.5,1.6/1,2.4/1.5}
\draw(0,\y)node[left]{$\ytext$};
%
\draw(2,-0.5)node[below]{$t/s$};
\foreach \x/\xtext in {0/0,1/0.1,2/0.2,3/0.3,4/0.4}
\draw(\x,0)node[below]{$\xtext$};
%
%Drehmoment
%
\visible<2->{\draw[line width = 1pt,color = blue](0,0)--(0,1.6)--(2,1.6);}
\visible<3->{\draw[line width = 1pt,color = red](2,1.6)--(2,0)--(4,0);}
%
%--------------------------------------------------------------------
%
\begin{scope}[xshift = 6.5cm]
%
\draw[dashed,ystep=0.8] (0,0) grid (4,2.4);
\draw[line width = 0.5pt] (0,0) rectangle (4,2.4);
%
\draw(-1,2)node[left,rotate = 90]{$\Omega/s^{-1}$};
\foreach \y/\ytext in {0/0,0.8/100,1.6/200,2.4/300}
\draw(0,\y)node[left]{$\ytext$};
%
\draw(2,-0.5)node[below]{$t/s$};
\foreach \x/\xtext in {0/0,1/0.1,2/0.2,3/0.3,4/0.4}
\draw(\x,0)node[below]{$\xtext$};
%
%Omega
%
\visible<2->{\draw[line width = 1pt,color = blue](0,0)--(2,1.6);}
\visible<3->{\draw[line width = 1pt,color = red](2,1.6)--(4,1.6);}
%
\end{scope}
%
%--------------------------------------------------------------------
%
\begin{scope}[yshift = -3.7cm]
%
\draw[dashed,ystep=0.8] (0,0) grid (4,2.4);
\draw[line width = 0.5pt] (0,0) rectangle (4,2.4);
%
\draw(-1,1.8)node[left,rotate = 90]{$\varphi/\text{rad}$};
\foreach \y/\ytext in {0/0,0.8/20,1.6/40,2.4/60}
\draw(0,\y)node[left]{$\ytext$};
%
\draw(2,-0.5)node[below]{$t/s$};
\foreach \x/\xtext in {0/0,1/0.1,2/0.2,3/0.3,4/0.4}
\draw(\x,0)node[below]{$\xtext$};
%
%\varphi
%
\visible<2->{\draw[line width = 1pt,color = blue,domain = 0:2]plot
(\x,{25*0.8*(\x*\x/100)});} %Skalierung der y-Achse auf 0.8 ^=20
\visible<3->{\draw[line width = 1pt,color =red](2,0.8)--(4,2.4);}
%
\end{scope}
%
\end{tikzpicture}
%
\end{frame}
%
%--------------------------------------------------------------------
%
\begin{frame}{Positionsvorgabe}
%
\begin{block}{Gegeben:}
%
\begin{columns}[T]
%
\column{0.35\textwidth}
Startposition $\varphi(t_0)=\varphi_0$
%
\column{0.6\textwidth}
Zielposition $\varphi(t_0+T_Z)=\varphi_Z$
%
\end{columns}
%
\end{block}
%
\begin{block}{Gesucht:}
%
\begin{columns}
%
\column{0.25\textwidth}
Bewegungsprofil:
%
\column{0.7\textwidth}
%
\begin{align}
%
\left. \begin{array}{l} \Omega(t) \\ M(t)\end{array}\right\} &\qquad \text{so
dass die Zielposition~$\varphi_Z$ erreicht wird}\nonumber
%
\end{align}
%
\end{columns}
%
\end{block}
%
Mögliche Vorgaben/Randbedingungen:
%
\begin{itemize}
  \item Vorgegebene Grenzwerte $\Omega_{\max}$, $\alpha_{\max}$, $M_{\max}$
  \item Randwerte $\Omega(t_0)$, $\Omega(t_0+T_Z)$, $\alpha(t_0)$,
  $\alpha(t_0+T_Z)$
  \item Dauer des Positioniervorgangs
  \begin{itemize}
    \item $T_Z$ vorgegeben
    \item $T_Z$ frei
  \end{itemize}
\end{itemize}
%
\end{frame}
%
%
%--------------------------------------------------------------------
%
\subsection[\thesection .\thesubsection \ 
Lageregelkreis]{Lageregelkreis}\label{sec:lageregelkreis}
%
%--------------------------------------------------------------------
%
\begin{frame}{Varianten der Positionierung}
%
\begin{block}{Absolutes Positionieren}
%
Die Zielposition bezieht sich immer auf den gleichen Referenzwert.
%
\end{block}
%
%
\begin{block}{Relatives Positionieren}
%
Die Zielposition bezieht sich immer auf die aktuelle Position.
%
\end{block}
%
\alert{Problem:} Aufgrund begrenzter  Genauigkeit pflanzen sich Fehler fort.
%
\begin{columns}[T]
%
\column{0.55\textwidth}
%
\begin{block}{Beispiel:}
%
Geberauflösung 1024 Striche / Umdr.\ 

Relatives Positionieren um jeweils 10 grad
%
\end{block}
%

\vspace{12pt}

\visible<2->{\alert{Abhilfe:} Korrektur durch Resteverwaltung}
%
\column{0.05\textwidth}

\column{0.35\textwidth}
%
\begin{tikzpicture}
%
%Wunsch
%
\foreach \x in
{0,10,20,30,40,50,60,70,80,90,100,110,120,130,140,150,160,170,180,190,200}
%
\draw[line width =0.5pt](0,0)--(\x:2);
%
%Ohne Korrektur
%
\foreach \x in
{0,9.84375,19.6875,29.53125,39.375,49.21875,59.0625,68.90625,78.75,88.59375,98.4375,108.28125,118.125,127.96875,137.8125,147.65625,157.5,167.34375,177.1875,187.03125,196.875}
%
\draw[blue,line width =0.5pt](0,0)--(\x:1.75);
%
%Mit Korrektur
%
\foreach \x in
{0,9.84375,19.6875,29.53125,39.726562,49.570312,59.765625,69.609375,
79.804688,89.648438,99.492188,109.6875,119.53125,129.72656,139.57031,
149.76562,159.60938,169.80469,179.64844,189.49219,199.6875}
%
\visible<2->{\draw[red,line width =0.5pt,dashed](0,0)--(\x:1.75);}
%
\draw[-latex] (0:2.2) arc (0:30:2.2)node[above right]{$\varphi$};
%
%Legende
%
\draw[line width =0.5pt]
(-1,-0.5)--(-0.5,-0.5)node[right]{$\varphi_{Z\text{Soll}}$}; 
\draw[blue,line width =0.5pt]
(-1,-1)--(-0.5,-1)node[right]{$\varphi_{Z\text{Ist}}$ unkorrigiert}; 
%
\visible<2->{\draw[red,line width
=0.5pt,dashed] (-1,-1.5)--(-0.5,-1.5)node[right]{$\varphi_{Z\text{Ist}}$
korrigiert};}
%
\end{tikzpicture}
%
\end{columns}
%
\end{frame}
w%
%--------------------------------------------------------------------
%
\section[\thesection \ Koordinierte Bewegungen]{Koordinierte
Bewegungen}\label{sec:koordinierteBewegungen}
%
%--------------------------------------------------------------------
%
\subsection[\thesection .\thesubsection \ 
Master--Slave--Anwendungen]{Master--Slave--Anwendungen}\label{sec:masterslave}
%
\begin{frame}{Aufgabenstellung}
%
\begin{block}{Synchronisation der Bewegung mehrerer Achsen:}
%

\vspace{-6pt}

%
\begin{itemize}
  \item Eine Masterachse
  \item Eine oder mehrere Slaveachsen
\end{itemize}
%
\end{block}
%

\vspace{6pt}

\begin{columns}[T]

\column{0.48\textwidth}
\begin{tikzpicture}
%
%--------------------------------------------------------------------
%
% Achse 1
%
\begin{scope}   %Achse1
%
\node[rotate = 90,rectangle,minimum width = 2.5cm,minimum height = 0.8cm,fill =
gray!80!](achse1) at (0,0) {Achse 1};

\begin{scope}

\tikzstyle{every node} = [shape = cylinder,rotate=190,draw,
                          cylinder uses custom fill,cylinder end fill = black!20!,cylinder body fill = black!50!]
%
\node[aspect = 0.85,inner xsep = 0.45cm,inner ysep = 0.4cm,draw] (motor)at
($(achse1.west)+(-0.3,-0.95)$){};
%

\node[aspect = 0.85,inner xsep = 0.15cm,inner ysep = 0.05cm,draw](welle) at
($(motor.east) + (0.27,0.05)$) {};

\end{scope}
%
%Motorleitung
\draw[line width = 2pt]
(achse1.west)--($(achse1.west)+(0,-0.25)$)--($(achse1.west)+(-0.27,-0.3)$)--($(achse1.west)+(-0.27,-0.55)$);
%
\draw[dashed](welle.east)--+(190:1.3)node[left]{$\varphi_1$, $\Omega_1$};
%
\draw[dashed,-latex] ($(welle.east)+(0,-0.5)+(190:0.7)$) arc
(-50:150:0.5);
%
\end{scope}

%
%--------------------------------------------------------------------
%
% Achse 2
%
\begin{scope}[xshift=0.85cm]   %Achse2
%
\node[rotate = 90,rectangle,minimum width = 2.5cm,minimum height = 0.8cm,fill =
gray!80!](achse2) at (0,0) {Achse 2};

\begin{scope}

\tikzstyle{every node} = [shape = cylinder,rotate=190,draw,
                          cylinder uses custom fill,cylinder end fill = black!20!,cylinder body fill = black!50!]

\node[aspect = 0.85,inner xsep = 0.45cm,inner ysep = 0.4cm,draw] (motor) at
($(achse2.west)+(0.34,-1.4)$){};
%

\node[aspect = 0.85,inner xsep = 0.15cm,inner ysep = 0.05cm,draw](welle) at
($(motor.east) + (0.27,0.05)$) {};
%
\end{scope}

\draw[dashed](welle.east)--+(190:3)node[left]{$\varphi_2$, $\Omega_2$};
\draw[dashed,-latex] ($(welle.east)+(0,-0.5)+(190:2.3)$)
arc(-50:150:0.5);

%Motorleitung
\draw[line width = 2pt]
(achse2.west)--($(achse2.west)+(0,-0.5)$)--($(achse2.west)+(0.37,-0.65)$)--($(achse2.west)+(0.37,-1)$);
%
\end{scope}
%
%--------------------------------------------------------------------
%
% Steuerung
%
\node[rotate = 90,rectangle,minimum width = 2.5cm,minimum height = 1.5cm,fill =
gray!80!](sps) at ($(achse1.north) + (-0.8,0)$){Steuerung};
%
\visible<2->{
\draw(sps.north)node[text width = 1.5cm,above left]{Virtueller Master};
\draw(sps.north)node[below left]{$\varphi_{VM}$, $\Omega_{VM}$};
}
%
\end{tikzpicture}

\column{0.47\textwidth}
%
\begin{block}{$\varphi_2$ , $\Omega_2 = f(\varphi_1)$ , $f(\Omega_1)$}
%
\begin{tabular}{ll}
  $\varphi_2 = K \varphi_1$ & Winkelsynchronlauf\\
  $\Omega_2 = K\Omega_1$    & Drehzahlsynchronlauf\\
  $\varphi_2 = f(\varphi_1)$ & Kurvenscheibe
\end{tabular}
%
\end{block}
%
\visible<2->{Masterachse kann auch durch \alert{\em Virtuellen Master} ersetzt
werden}
%
\end{columns}
%
\end{frame}
%
%--------------------------------------------------------------------
%
\begin{frame}{Drehzahlsynchronlauf}
%
Beispiel: Fliegende Säge
%
\begin{itemize}
  \item Master fährt mit konstanter Drehzahl
  %
  \item Slave 
  %
  \begin{enumerate}
    \item Ist im Stillstand in der Position~$X_0$
    \item Beschleunigt auf die Masterdrehzahl
    \item Fährt mit der Masterdrehzahl
    \item Fährt zur Position $X_0$
  \end{enumerate}
  %
  \item Weitere unsynchronisierte Achsen
  %
  \begin{itemize}
    %
    \item Vorschub der Säge von Position $Y_0$ nach $Y_1$
    \item Antrieb der Säge
    %
  \end{itemize}
  %
\end{itemize}
%
\begin{block}{Beispiel Applikationsdaten}
%
\begin{tabular}{llp{0.25cm}ll}
%
Band                &         &&Säge\\[0.25em]
\cline{1-2} \cline{4-5}
%
\\[-0.75em]

%
Geschwindigkeit:     &20m/min  &&Vorschubgeschwindigkeit:&10m/min\\
Breite:              &20cm     &&Beschleunigung X (Slave):&5m/s$^{2}$\\
Länge:               &1m       &&Beschleunigung Y:&6m/s$^{2}$
%
\end{tabular}
%
\end{block}
%
\end{frame}
%
%
%--------------------------------------------------------------------
%
\subsection[\thesection .\thesubsection \ 
Interpolierte Bewegungen]{Interpolierte Bewegungen}\label{sec:interpolation}
%
\begin{frame}{Interpolierte Bewegungen}
%
\begin{block}{Aufgabenstellung}
%
Verfahren einer beliebigen durch die Anwendung vorgegebenen Bahn in der Ebene
/ im Raum
%
\begin{itemize}
  \item Aufteilung der Bewegung in der Ebene / im Raum auf die einzelnen Achsen
  \item Zerlegung der Bahn in Stützstellen zu diskreten Zeitpunkten
  \item Interpolation zwischen den Stützstellen
\end{itemize}
%
\end{block}
%

\vspace{12pt}

Typische Anwendungsfelder
%
\begin{itemize}
  \item Robotik
  \item Werkzeugmaschinen
\end{itemize}
%


\vspace{36pt}


\end{frame}
%
%--------------------------------------------------------------------
%
\begin{frame}{Darstellung einer zweidimensionalen Bahn}
%
\vspace{-3pt}
%
\begin{block}{Implizite Darstellung}
%
\begin{columns}
%
\column{0.2\textwidth}
%
\begin{align}
%
0 = F(X,Y)\nonumber
%
\end{align}
%
\column{0.25\textwidth}
%
Beispiel: Kreis

\vspace{-18pt}

\begin{align}
%
X^2+Y^2 &= 1 \nonumber 
%
\end{align}
%
\column{0.25\textwidth}
%
\begin{tikzpicture}
%
\draw[fill=white,draw] (-0.6,-0.6)rectangle(0.6,0.6);
\draw[thick,black] (0,0) circle(0.3cm);
\draw[-latex,black](-0.5,0)--(0.5,0)node[below]{$\scriptstyle{X}$};
\draw[-latex,black](0,-0.5)--(0,0.5)node[left]{$\scriptstyle{Y}$};
%
\end{tikzpicture}
%
\column{0.25\textwidth}
%
%
\end{columns}
%
\end{block}

\vspace{-6pt}

\begin{block}{Explizite Darstellung}
%
\begin{columns}
%
\column{0.2\textwidth}
%
\begin{align}
%
X = F_1(Y)\nonumber\\
Y = F_2(X)\nonumber
%
\end{align}
%
\column{0.25\textwidth}
%
Beispiel: Kreis

\vspace{-18pt}

\begin{align}
%
X &= \sqrt{1-Y^2} \nonumber\\
Y &= \sqrt{1-X^2} \nonumber 
%
\end{align}
%
\column{0.25\textwidth}
%
\begin{tikzpicture}
%
\draw[fill=white,draw] (-0.6,-0.6)rectangle(0.6,0.6);
\draw[thick,black] (0,0) circle(0.3cm);
\draw[-latex,black](-0.5,0)--(0.5,0)node[below]{$\scriptstyle{X}$};
\draw[-latex,black](0,-0.5)--(0,0.5)node[left]{$\scriptstyle{Y}$};
%
\end{tikzpicture}
%
\column{0.25\textwidth}
%

%
\end{columns}
%
\end{block}
%

\vspace{-6pt}

%
\begin{block}{Parametrische Darstellung}
%
\begin{columns}
%
\column{0.2\textwidth}
%
\begin{align}
%
X = F_1(u)\nonumber\\
Z = F_2(u)\nonumber
%
\end{align}
%
\column{0.25\textwidth}
%
Beispiel: Kreis
%

\vspace{-18pt}

\begin{align}
%
X &= \sin{u} \nonumber\\
Z &= \cos{u} \nonumber 
%
\end{align}
%
\column{0.15\textwidth}
%
\begin{tikzpicture}
%
\draw[fill=white,draw] (-0.6,-0.6)rectangle(0.6,0.6);
\draw[thick,black] (0,0) circle(0.3cm);
\draw[-latex,black](-0.5,0)--(0.5,0)node[below]{$\scriptstyle{X}$};
\draw[-latex,black](0,-0.5)--(0,0.5)node[left]{$\scriptstyle{Y}$};
%
\end{tikzpicture}
%
\column{0.2\textwidth}
%
Beispiel: Kurve

\vspace{-18pt}

\begin{align}
%
X &= u\sin{u} \nonumber\\
Y &= u\cos{u} \nonumber 
%
\end{align}
%
\column{0.1\textwidth}
%
\begin{tikzpicture}
%
\draw[fill=white,draw] (-0.3,-0.6)rectangle(0.6,0.6);
\draw[-latex,black](-0.2,0)--(0.5,0)node[below]{$\scriptstyle{X}$};
\draw[-latex,black](0,-0.5)--(0,0.5)node[left]{$\scriptstyle{Y}$};
%
\draw[thick,black,scale=0.15,domain=-2.9:2.9,smooth,variable=\t]
plot ({\t*sin(\t r)},{\t*cos(\t r)});
%
\end{tikzpicture}
%
\end{columns}
%
\end{block}
%
\end{frame}
%
%--------------------------------------------------------------------
%
\begin{frame}{Satzübergänge}
%
Übergang zwischen den Teilstücken des interpolierten Bewegungsprofils
%
\begin{enumerate}
  \item Standardverfahren
\end{enumerate}
%
\begin{columns}[T]
%
\column{0.1\textwidth}
%
\column{0.25\textwidth}
%
\begin{itemize}
  \item Übergang mit {\em Genauhalt}
\end{itemize}
%
\column{0.6\textwidth}
\begin{tikzpicture}
%
\coordinate (start) at (0,1);
\coordinate (ecke) at (1,1);
\coordinate (semitte) at ($(start)!0.65!(ecke)$);
\coordinate (ende) at (1,0);
\coordinate (eemitte) at ($(ecke)!0.65!(ende)$);
%
\draw[-latex,thick](start)--(semitte);
\draw[-latex,thick](ecke)--(eemitte);
\draw[thick](start)--(ecke)--(ende);
%
\begin{scope}[xshift = 3cm]
%
\draw[-latex](-0.1,0)--(3.5,0);
\draw[-latex](0,-0.1)--(0,1)node[left]{$V_X,V_Y$};
%
\draw[thick,blue](0,0.6)--(1.25,0.6)--(1.5,0);
\draw[thick,dashed,red](1.5,0)--(1.75,0.6)--(3,0.6);
%
\end{scope}
%
\end{tikzpicture}
%
\end{columns}
%
\vspace{12pt}
%
\begin{columns}[T]
%
\column{0.1\textwidth}
%
\column{0.25\textwidth}
%
\begin{itemize}
  \item Übergang mit {\em Überschleifen}
\end{itemize}
%
\column{0.6\textwidth}
\input{./Bilder/satzuebergangueberschleifen.tex}
%
\end{columns}
%
\vspace{12pt}
%
\begin{columns}[T]
%
\column{0.1\textwidth}
%
\column{0.25\textwidth}
%
\begin{itemize}
  \item Übergang mit {\em Grenzgeschwindigkeit}
\end{itemize}
%
\column{0.6\textwidth}
\input{./Bilder/satzuebergangvmin.tex}
%
\end{columns}
%

\vspace{6pt}

Satzübergänge ohne Genauhalt
%
\begin{itemize}
  \item[$\Rightarrow$] Eckenverrundung aufgrund der Maschinendynamik / der
  Schleppfehler
\end{itemize} 
%
\end{frame}
%
%
%--------------------------------------------------------------------
%
\begin{frame}{Satzübergänge}
%
\begin{enumerate}
  \setcounter{enumi}{0}
  \item Standardverfahren (Fortsetzung)
\end{enumerate}
%
\vspace{6pt}
%
\begin{columns}
%
\column{0.1\textwidth}
%
\column{0.85\textwidth}
%

Beispiel: Geradeninterpolation zwischen drei Punkten

\vspace{6pt}

%
\begin{tikzpicture}
%
%Positionen/Kontur
%
\begin{scope}[xscale = 0.15,yscale = 0.15]
%
\draw[-latex] (-1,0)node[left]{$Y_0$}--(25,0)node[below]{$X$};
\draw[-latex] (0,-1)node[below]{$X_0$}--(0,20)node[left]{$Y$};
%
\draw[thick](0,0)--(7,7)--(20,14.5);
%
\draw[dashed](0,7)node[left]{$Y_1$}--(7,7)--(7,0)node[below]{$X_1$};
\draw[dashed](0,14.5)node[left]{$Y_2$}--(20,14.5)--(20,0)node[below]{$X_2$};
%
\end{scope}
%
\begin{scope}[xshift = 4.5cm,xscale = 0.01,yscale = 0.01]
%
\draw[-latex] (-10,0)--(460,0)node[below]{$t$};
\draw[-latex] (0,-10)--(0,125)node[left]{$V_X$};
%
\draw[thick](0,0)--(84,84)--(168,0)node[below]{$t_1$}--(294,103)--(420,0)node[below]{$t_2$};
%
\draw[dashed](84,0)--(84,230);
\draw[dashed](168,0)--(168,150);
\draw[dashed](294,0)--(294,200);
\draw[dashed](420,0)--(420,150);
%
\end{scope}
%
\begin{scope}[xshift = 4.5cm,yshift = 1.5cm,xscale = 0.01,yscale = 0.01]
%
\draw[-latex] (-10,0)--(460,0)node[below]{$t$};
\draw[-latex] (0,-10)--(0,125)node[left]{$V_Y$};
%
\draw[thick](0,0)--(84,84)--(168,0)--(294,59.5)--(420,0);
%
\end{scope}
%
\end{tikzpicture}
%

\vspace{6pt}
%

Die gesamte Bahn setzt sich aus einzelnen $R$--$R$--Segmenten zusammen

%
\end{columns}
%
\vspace{12pt}
%
\end{frame}
%
%--------------------------------------------------------------------
%
\begin{frame}{Satzübergänge}
%
Übergang zwischen den Teilstücken des interpolierten Bewegungsprofils
%
\begin{enumerate}
  \setcounter{enumi}{1}
  \item Übergang mit {\em Look--Ahead}
\end{enumerate}
%
\begin{columns}
%
\column{0.1\textwidth}
%
\column{0.85\textwidth}
%

Forsetzung Beispiel: Geradeninterpolation zwischen drei Punkten

\vspace{6pt}

%
\begin{tikzpicture}
%
%Positionen/Kontur
%
\begin{scope}[xscale = 0.15,yscale = 0.15]
%
\draw[-latex] (-1,0)node[left]{$Y_0$}--(25,0)node[below]{$X$};
\draw[-latex] (0,-1)node[below]{$X_0$}--(0,20)node[left]{$Y$};
%
\draw[thick](0,0)--(7,7)--(20,14.5);
%
\draw[dashed](0,7)node[left]{$Y_1$}--(7,7)--(7,0)node[below]{$X_1$};
\draw[dashed](0,14.5)node[left]{$Y_2$}--(20,14.5)--(20,0)node[below]{$X_2$};
%
\end{scope}
%
%Geschwindigkeiten x
%
\begin{scope}[xshift = 4.5cm,xscale = 0.01,yscale = 0.01]
%
\draw[-latex] (-10,0)--(460,0)node[below]{$t$};
\draw[-latex] (0,-10)--(0,125)node[left]{$V_X$};
%
\draw[thick](0,0)--(84,84)--(126,84)--(140,103)--(192,103)--(295,0)node[below]{$t_2$};
%
\draw[dashed](84,0)--(84,230);
\draw[dashed](126,0)node[below]{$t_1$}--(126,230);
\draw[dashed](140,0)--(140,200);
\draw[dashed](192,0)--(192,200);
\draw[dashed](295,0)--(295,150);

%
\end{scope}
%
%Geschwindigkeiten y
%
\begin{scope}[xshift = 4.5cm,yshift = 1.5cm,xscale = 0.01,yscale = 0.01]
%
\draw[-latex] (-10,0)--(460,0)node[below]{$t$};
\draw[-latex] (0,-10)--(0,125)node[left]{$V_Y$};
%
\draw[thick](0,0)--(84,84)--(126,84)--(140,59.5)--(192,59.5)--(295,0);
%
\end{scope}
%
\end{tikzpicture}
%

\vspace{6pt}
%

\begin{itemize}
  \item Keine R--R Bewegungen mehr
  \item Geringere Geschwindigkeitsänderungen
\end{itemize}

%
\end{columns}
%
\vspace{12pt}
%
\end{frame}
%

%
  
\end{document}