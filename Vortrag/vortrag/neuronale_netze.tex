\section[\thesection \  Künstliche Neuronale Netze]{Künstliche Neuronale Netze}\label{sec:nn}

% Frame 1
\begin{frame}{Machine Learning}
    \vspace{0.5cm}
    %\begin{block}
        erkennung von mustern/zusammenhängen in großer 
        datenmenge ohne expliziert programmiert zusein
        \begin{figure}[h]
            \centering
            \def\svgwidth{0.8\columnwidth}
            \tiny
            
\tikzset{
    decision/.style={
        diamond,
        draw,
        text width=4em,
        text badly centered,
        inner sep=-1pt,
        node distance=8em
    },
    block/.style={
        rectangle,
        draw,
        text width=6em,
        %minimum widhth=6em,
        minimum height=5em,
        text centered,
        node distance=20em
    },
    arrow/.style={
        draw,
        >=latex,
        ->
    },
    textfeld/.style={
        %draw,
        text centered,
        node distance=1.5em
    }
}


\begin{tikzpicture}

    
    \node (system) [block] {Klassisches\\Programm};
    \node (system2) [block, right of=system] {Machine Learning\\Programm};

    \node [textfeld, left=of system.162] (inputs) {Daten};
    \node [textfeld, left=of system.198] (regeln) {Regeln};
    \node [textfeld, right=of system] (output) {Ausgaben};

    \node [textfeld, left=of system2.162] (inputs2) {Daten};
    \node [textfeld, left=of system2.198] (output2) {Ausgaben};
    \node [textfeld, right=of system2] (regeln2) {Regeln};
    
    \draw[arrow] (inputs) -- (system.162);
    \draw[arrow] (regeln) -- (system.198);
    \draw[arrow] (system) -- (output);
    
    \draw[arrow] (inputs2) -- (system2.162);
    \draw[arrow] (output2) -- (system2.198);
    \draw[arrow] (system2) -- (regeln2);
    

\end{tikzpicture}

        \end{figure}
    %\end{block}

    \begin{block}{Neuronale Netze}
        \begin{columns}[T]
            \column{0.3\columnwidth}
            Neuronale netze \dots
            \column{0.7\columnwidth}
            \begin{figure}[htb]
                \centering
                \tiny
                \begin{neuralnetwork}[height=1]
    \newcommand{\nodetextclear}[2]{}
    \newcommand{\nodetexth}[2]{$h_#2$}
    \newcommand{\nodetextx}[2]{$x_#2$}
    \newcommand{\nodetexty}[2]{$y_#2$}
    \inputlayer[count=3, bias=false, title=Input\\layer, text=\nodetextx]
    \hiddenlayer[count=4, bias=false, title=Hidden\\layer, text=\nodetexth] \linklayers
    \outputlayer[count=2, title=Output\\layer, text=\nodetexty] \linklayers
\end{neuralnetwork}
            \end{figure}         
        \end{columns}
    \end{block}    
\end{frame}



% Frame 2
\begin{frame}{Training vs Inferenz}
    \begin{columns}[T]
        \column{0.5\columnwidth}
        Training
        \begin{itemize}
            \item variable parameter
            \item gelablte input daten
        \end{itemize}
        \column{0.5\columnwidth}
        Inferenz
        \begin{itemize}
            \item fixe parameter
            \item unbekannte input daten
        \end{itemize}
    \end{columns}
\end{frame}


